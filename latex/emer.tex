% !iTeXMac(input): POH.tex
\chapter{EMERGENCY PROCEDURES}
\vspace{\minitocspacebefore}
\minitoc
\cleardoublepage
\section{INTRODUCTION}
Section 3 provides procedures to address emergencies that may occur.   Should an emergency arise, the basic guidelines in this section should be considered and applied as necessary to correct the problem.

Some emergency procedures receive additional discussion in the Amplified Procedures section which follows the Emergency Checklists.

\section{AIRSPEEDS FOR EMERGENCY OPERATION}
%(\textcolor{red}{Red text indicates provisional information, based
%on the RV-8A Aircraft Performance Report published by the CAFE Foundation. This data has not yet been validated by flight test.})

\begin{quote}
Engine Failure After Takeoff\\
\newlength\emertab
\setlength\emertab{0.2in}
%\hspace{\emertab}Flaps UP\dotfill \textcolor{red}{XX} KIAS\\
\hspace*{\emertab}Flaps UP\dotfill 115 KIAS\\
\hspace*{\emertab}Flaps DOWN\dotfill 80 KIAS\\
Manoeuvring Speed\\
\hspace*{\emertab}1550 lb (703.1 kg) or greater\dotfill 120 KIAS\\
\hspace*{\emertab}1300 lb (589.7 kg)\dotfill 110 KIAS\\
Maximum Glide\\
\hspace*{\emertab}\theMTOW \space lb (\theMTOWkg .\theMTOWkgdecimal \space kg)\dotfill 115 KIAS\\
\hspace*{\emertab}1600 lb (725.7 kg)\dotfill 105 KIAS\\
\hspace*{\emertab}1300 lb (589.7 kg)\dotfill 95 KIAS\\
Precautionary Landing With Engine Power\dotfill 70 KIAS\\
Landing Without Engine Power\\
\hspace*{\emertab}Flaps UP\dotfill 115 KIAS\\
\hspace*{\emertab}Flaps DOWN\dotfill 80 KIAS
\end{quote}

\cleardoublepage
% !iTeXMac(input): POH.tex
\changepage{\checklistTextHeight}{\checklistTextWidth}{\checklistEvenSideMargin}{\checklistOddSideMargin}{}{\checklistTopMargin}{0pt}{0pt}{0pt}% original settings, with no room for header or footer

%% next three lines show a mini page diagram with margins
%\currentpage
%\setlayoutscale{0.5}
%\pagedesign

\small
%\footnotesize
%\addcontentsline{toc}{section}{\numberline { }EMERGENCY CHECKLIST}
%\addcontentsline{toc}{section}{\numberline EMERGENCY CHECKLIST}
\addcontentsline{toc}{section}{EMERGENCY CHECKLISTS}
%\mtcaddsection[EMERGENCY CHECKLIST]
\chead{\Large EMERGENCY CHECKLISTS} % puts "EMERGENCY CHECKLIST" in the header.
% \ohead{DRAFT --- \thistime \ \cdndate\today}
\ohead{Created --- \thistime \ \cdndate\today}

\begin{multicols}{2}

%\section{EMERGENCY CHECKLISTS} % leave this line commented out, 
%as it destroys the visual flow of the checklist page.
% Find a way to get "EMERGENCY CHECKLISTS" in the header instead.
\subsection*{ENGINE FAILURES}
%\renewcommand\sectionmark{\markright {EMERGENCY CHECKLISTS}}


\subsubsection*{\fcolorbox{black}{red}{ENGINE FAILURE DURING TAKEOFF RUN}}

\begin{enumerate*}
  \item Throttle \dotfill IDLE
  \item Brakes \dotfill APPLY
  \item Flaps \dotfill RETRACT
  \item \emph{If insufficient runway remains:}
  \begin{itemize*}
%    \item Mixture \dotfill IDLE CUT-OFF
    \item Fuel Selector \dotfill OFF
    \item Ignition Switches (Both) \dotfill OFF
    \item BATT/ALT \dotfill OFF
    \end{itemize*}
  \end{enumerate*}

\subsubsection*{\fcolorbox{black}{red}{ENGINE FAILURE IMMEDIATELY AFTER TAKEOFF}}

\begin{enumerate*}
\item Airspeed \raggedright \dotfill 115 KIAS (Flaps UP)\\\hfill80 KIAS (Flaps DOWN)
\item Mixture \dotfill IDLE CUT-OFF
\item Prop \dotfill MIN RPM
\item Fuel Selector \dotfill OFF
\item Ignition Switches (Both) \dotfill OFF
\item Flaps \dotfill AS REQUIRED
\item BATT/ALT \dotfill OFF
\end{enumerate*}

\subsubsection*{\fcolorbox{black}{red}{ENGINE FAILURE DURING FLIGHT}}

\begin{enumerate*}
\item Airspeed \dotfill 115 KIAS
\item Fuel Selector \dotfill SWITCH TANKS
\item Boost Pump \dotfill ON
\item Mixture \dotfill RICH
\item Alternate Air \dotfill ON
% \item Left Ignition Switch \dotfill ON, OFF, ON
% \item Right Ignition Switch \dotfill ON, OFF, ON
\item IGNITION, ELEC \dotfill ON, OFF, ON
\item IGNITION, MAG \dotfill ON, OFF, ON
\item Starter \dotfill START (if propeller is stopped)
\item Transponder\dotfill 7700
\end{enumerate*}

\subsubsection*{\fcolorbox{black}{yellow}{ROUGH RUNNING ENGINE}}

\begin{enumerate*}
\item Mixture \dotfill ADJUST
\item Throttle \dotfill ADJUST
\item Boost Pump \dotfill ON
\item Fuel Selector \dotfill CHANGE TANKS
\item Alternate Air \dotfill ON
\suspend{enumerate*}
\begin{Note}[CAUTION]
If engine quits when ignition selected OFF, select the mixture to ICO, wait 10 seconds, then select the ignition back ON.
Advance mixture slowly until engine restarts.
\end{Note}
\resume{enumerate*}
% \item Left Ignition Switch \dotfill ON, OFF, ON
% \item Right Ignition Switch \dotfill ON, OFF, ON
\item IGNITION, ELEC \dotfill ON, OFF, ON
\item IGNITION, MAG \dotfill ON, OFF, ON
\item Prepare for power off landing
\end{enumerate*}

\subsubsection*{\fcolorbox{black}{yellow}{HIGH OIL TEMPERATURE}}

\begin{enumerate*}
\item Oil Cooler Door \dotfill OPEN
\item Oil temperature and pressure \dotfill MONITOR
\end{enumerate*}

\subsubsection*{\fcolorbox{black}{yellow}{LOW FUEL PRESSURE}}

\begin{enumerate*}
\item Boost Pump \dotfill ON
\item Fuel Selector \dotfill CHANGE TANKS
\end{enumerate*}

\subsubsection*{\fcolorbox{black}{yellow}{LOW OIL PRESSURE}}

\begin{enumerate*}
\item OIL PRESS light \dotfill CHECK
\item EIS Oil Pressure Indication \dotfill CHECK
\suspend{enumerate*}
%\begin{Note}
%EIS oil pressure and OIL PRESS light use different transducers. In
%case of discrepancy, suspect false warning.
%\end{Note}
\begin{Note}
In case of discrepancy between indications, suspect false warning.
\end{Note}
\resume{enumerate*}
\item \emph{If both indications confirm low oil pressure:}
\begin{itemize*}
  \item Land ASAP
  \end{itemize*}
\end{enumerate*}

\subsubsection*{\fcolorbox{black}{yellow}{AIR-START}}

\begin{enumerate*}
  \item Throttle \dotfill 1/4 OPEN
  \item Prop \dotfill MAX RPM
  \item Fuel Pressure \dotfill CHECK
  \item \emph{If fuel pressure less than 14 psi:}
  \begin{itemize*}
    \item Boost Pump \dotfill ON
    \item Fuel Selector \dotfill CHANGE TANKS
    \end{itemize*}
  \item Ignition Switches\dotfill BOTH ON
  \item Mixture \dotfill IDLE CUT-OFF
%  \item Mixture \raggedright \dotfill ADVANCE SLOWLY\\\hfill UNTIL ENGINE STARTS
  \item Mixture \dotfill ADVANCE SLOWLY UNTIL ENGINE STARTS
  \item \emph{When engine starts:}
  \begin{itemize*}
    \item Throttle \dotfill A/R
    \item Boost Pump \dotfill OFF
    \end{itemize*}
  
  \end{enumerate*}

\subsection*{FORCED LANDINGS}
\subsubsection*{\fcolorbox{black}{red}{EMERGENCY LANDING WITHOUT POWER}}

\begin{enumerate*}
\item Airspeed \raggedright \dotfill 115 KIAS (Flaps UP)\\\hfill80 KIAS (Flaps DOWN)
\item Throttle \dotfill CLOSED
\item Mixture \dotfill IDLE CUT-OFF
\item Prop \dotfill MIN RPM
\item Fuel Selector \dotfill OFF
\item Ignition Switches (Both) \dotfill OFF
\item Radio \dotfill TRANSMIT MAYDAY
\item Flaps \dotfill A/R
\item BATT/ALT \dotfill OFF
\end{enumerate*}

\subsubsection*{\fcolorbox{black}{yellow}{EMERGENCY LANDING WITH POWER}}

\begin{enumerate*}
\item Radio \dotfill TRANSMIT MAYDAY
\item Airspeed \dotfill 85 KIAS
\item Flaps \dotfill 50\%
\item Selected Field \dotfill FLY OVER
\item Flaps \dotfill FULL
\item Airspeed \dotfill 70 KIAS
\item BATT/ALT \dotfill OFF
\item Ignition Switches (Both) \dotfill OFF (after touchdown)
\end{enumerate*}

%\clearpage

\subsubsection*{\fcolorbox{black}{red}{DITCHING}}

\begin{enumerate*}
\item Radio \dotfill TRANSMIT MAYDAY
\item Flaps \dotfill FULL
\item Airspeed \dotfill 70 KIAS
\item Power \dotfill 300 FT/MIN DESCENT
%\item Approach - High Wind \dotfill INTO WIND
%\item - Light Wind \dotfill PARALLEL TO SWELLS
\item Approach \raggedright \dotfill High Wind - INTO WIND\\\hfill Light Wind - PARALLEL TO SWELLS
\item Face \dotfill CUSHION with folded coat
\item \emph{Leaving Aircraft}

\begin{itemize*}
  \item Seat Belts \dotfill RELEASE
  \item Canopy \dotfill OPEN
  \item Aircraft \dotfill EXIT
  \item Life Jacket \dotfill INFLATE 
  \end{itemize*}
\end{enumerate*}

\subsection*{FIRES}
\subsubsection*{\fcolorbox{black}{red}{FIRE DURING START ON GROUND}}
\begin{enumerate*}
  % \item Boost Pump \dotfill OFF
  \item Mixture \dotfill IDLE CUT-OFF
  \item Fuel Selector \dotfill OFF
  \item Starter \dotfill OFF
  \item BATT/ALT \dotfill OFF
  \item Fire Extinguisher\dotfill OBTAIN
  \item Aircraft\dotfill EVACUATE
  \item Fire\dotfill EXTINGUISH
  \end{enumerate*}

\subsubsection*{\fcolorbox{black}{red}{ENGINE FIRE IN FLIGHT}}
\begin{enumerate*}
\item Mixture \dotfill IDLE CUT-OFF
\item Fuel Selector \dotfill OFF
\item Boost Pump \dotfill OFF
\item Cabin Heat and Air \dotfill OFF
\item ESS BUS FEED \dotfill EMER
\item BATT/ALT \dotfill OFF
\item Forced Landing \dotfill COMPLETE
\end{enumerate*}

\subsubsection*{\fcolorbox{black}{red}{ELECTRICAL FIRE IN FLIGHT}}

\begin{enumerate*}
\item BATT/ALT \dotfill OFF
\item STBY ALT \dotfill OFF
\item ESS BUS FEED \dotfill NORM
\item EFIS auto-shutdown \dotfill OVERRIDE (if EFIS required)
\item Avionics \dotfill OFF
\item All Other Switches (except Mag and TURN COORD)\dotfill OFF
\item Vents/ Cabin Air/ Heat \dotfill CLOSED
\item Fire Extinguisher \dotfill ACTIVATE
\item Cabin \dotfill VENTILATE
\end{enumerate*}

\begin{Note}[WARNING]
  Ventilate the cockpit ASAP after discharging the fire extinguisher.
  \end{Note}
  
\subsubsection*{\fcolorbox{black}{red}{CABIN FIRE}}

\begin{enumerate*}
\item BATT/ALT \dotfill OFF
\item STBY ALT \dotfill OFF
\item ESS BUS FEED \dotfill NORM
\item EFIS auto-shutdown \dotfill OVERRIDE (if EFIS required)
\item Vents/ Cabin Heat \dotfill CLOSED
\item Fire Extinguisher \dotfill ACTIVATE
\item Cabin \dotfill VENTILATE
\end{enumerate*}

\begin{Note}[WARNING]
  Ventilate the cockpit ASAP after discharging the fire extinguisher.
  \end{Note}
  
\subsubsection*{\fcolorbox{black}{red}{WING FIRE}}

\begin{enumerate*}
\item Nav/Strobe Light Switch \dotfill OFF
\item Landing \& Taxi Lights \dotfill OFF
\item PITOT HEAT \dotfill OFF
\end{enumerate*}

\begin{Note}
  Perform a sideslip to keep the flames away from the fuel tank and cockpit, and land as soon as possible.
  \end{Note}

\subsection*{OTHER}
\subsubsection*{\fcolorbox{black}{yellow}{STATIC SOURCE BLOCKAGE}}
\begin{enumerate*}
  \item Vents/ Cabin Air/ Heat \dotfill CLOSED
  \item Alternate Static Source Valve\dotfill  OPEN
  \item Airspeed and altitude corrections\dotfill  APPLY
  \end{enumerate*}

\columnbreak
\subsubsection*{\fcolorbox{black}{yellow}{ALTERNATOR FAILURE}}

\begin{enumerate*}
\item BATT/ALT \dotfill BATT, then BATT + ALT
\item \emph{If failure continues:}

\begin{itemize*}
\item ESS BUS FEED \dotfill EMER
\item \emph{If instrument lighting required:}

\begin{itemize*}
\item[\textbullet] Map Light \dotfill ON
\end{itemize*}
\item Radios \dotfill SELECT COM 1 (GNS 430)
\item BATT/ALT \dotfill OFF
\item STBY ALT \dotfill ON
\item EFIS BU \dotfill ON
\item Voltage \dotfill MONITOR
%\item \emph{Prior to landing, if Main Bus powered equipment is required:}
%\begin{itemize*}
%\item BATT/ALT \dotfill ON
%\end{itemize*}
\item The following Main Bus powered equipment is inoperative:

\begin{tabular}{ll}
COM 2&Landing and Taxi Lights\\
Intercom&CDI Lighting\\
Pitot Heat&Eng. Instrument Lighting\\
Flaps&Strobe Lights\\
Boost Pump&Position Lights\\
Analog Tach and MP&Low Oil Press. Light\\
Autopilot&Defrost Fan\\
\end{tabular}
\item \emph{Prior to landing, if equipment powered from Main Bus is required:}
\begin{itemize*}
\item BATT/ALT \dotfill ON
\end{itemize*}
\end{itemize*}
\end{enumerate*}

\subsubsection*{\fcolorbox{black}{yellow}{STARTER ENGAGED LIGHT ILLUMINATED}}
\begin{enumerate*}
  \item BATT/ALT \dotfill OFF
  \item Aircraft \dotfill LAND ASAP
  \end{enumerate*}

\subsubsection*{\fcolorbox{black}{red}{AUTOPILOT MALFUNCTION}}
\begin{enumerate*}
  \item Trim/Autopilot Cut-out \dotfill PRESS AND HOLD
  \item Autopilot Power Switch (Autopilot Control Head)\dotfill OFF
  \item WING LVLR Switch (Right Console)\dotfill OFF
  \item Trim/Autopilot Cut-out \dotfill RELEASE
%  \item \emph{If failure continues:}
%  \begin{itemize*}
%    \item Wing Leveler/Control Stick ``pip'' Pin \dotfill PULL
%    \end{itemize*}
  \end{enumerate*}

\subsubsection*{\fcolorbox{black}{red}{RUNAWAY TRIM}}
\begin{enumerate*}
  \item Trim/Autopilot Cut-out \dotfill PRESS AND HOLD
  \item TRIM Switch \dotfill OFF
  \item Trim/Autopilot Cut-out \dotfill RELEASE
  \end{enumerate*}

\subsubsection*{\fcolorbox{black}{red}{AIRBORNE EGRESS}}

\begin{enumerate*}
\item Helmet Jacks \dotfill UNPLUG
\item Canopy Jettison Pins \dotfill REMOVE
\item Canopy \dotfill UNLATCH \& PULL AFT
\item Canopy \dotfill PUSH UP (lower head)
\item Seat Belt \dotfill RELEASE
\item Aircraft \dotfill EGRESS
\end{enumerate*}

\subsubsection*{\fcolorbox{black}{red}{CO MONITOR ALARM}}
\begin{enumerate*}
\item Power \dotfill REDUCE
\item Mixture \dotfill WELL LEAN OF PEAK
\item Cabin Heat (Both) \dotfill CLOSED
\item Fresh Air Vents (Both) \dotfill OPEN
\item CO Monitor \dotfill MONITOR READINGS
\item \emph{If red light on CO Monitor remains illuminated:}

  \begin{itemize*}
    \item Airspeed \dotfill 80 KT MAX
    \item Canopy \dotfill OPEN SLIGHTLY
    \end{itemize*}
\end{enumerate*}
\end{multicols}


%\cleardoublepage
\normalsize
\cleardoublepage
\ohead{\leftmark} % put Section name back in outer header
\chead{} % remove center header
\changepage{\checklistTextHeight*-1}{\checklistTextWidth*-1}{\checklistEvenSideMargin*-1}{\checklistOddSideMargin*-1}{}{\checklistTopMargin*-1}{0pt}{0pt}{0pt}% original settings, with no room for header or footer


%\cleardoublepage

\section{AMPLIFIED EMERGENCY PROCEDURES}

\subsection{ENGINE FAILURES}
\subsubsection{ENGINE POWER LOSS DURING TAKEOFF}

If an engine failure occurs during the takeoff run, the most important
thing to do is stop the aircraft on the remaining runway. Those extra
items on the checklist will provide added safety.

The first response to an engine failure after takeoff is to promptly
lower the nose to maintain airspeed and to establish a glide.  Pulling the prop control full aft may significantly reduce windmilling drag if oil pressure
is available. In most
cases, the landing should be made straight ahead with only small changes
in direction to avoid obstructions. A turn back to the runway should
not be attempted below 1,000 ft AGL, as the aircraft must be turned
through more than 180\textdegree \ to align with the runway. The checklist procedures
assume that sufficient time is available to secure the fuel and ignition
systems prior to touchdown. Flaps should normally be fully extended
prior to touchdown.

\subsubsection{ENGINE POWER LOSS IN FLIGHT}

Complete power loss is usually due to fuel interruption, if this is
so, power will be restored when fuel flow is itself restored. The
first action is to trim for best glide 95 - 115 KIAS, depending on
weight, and decide if there is time to attempt restart or whether to immediately
prepare for an emergency ``Power Off'' landing.

Select throttle CLOSED and prop control FULL AFT to reduce drag from the windmilling prop.  The prop will continue to windmill, even if the speed is slowed to the stall with flaps UP, unless the engine has sustained internal damage.  While it is possible to get the prop to stop if the aircraft is slowed just above the stall for about 2 minutes with flaps DOWN, throttle FULL OPEN, significant altitude is lost during this time.  Flight testing showed that by the time the prop stops, the aircraft will be about 900 ft lower than it would have been if it had been gliding with the prop windmilling.  Over 10,000 ft of further descent will be required before the improved glide performance with prop stopped would allow the aircraft to better the performance with prop windmilling.  No attempt should be made to stop the prop unless the aircraft is at least 15,000 ft AGL.  If the prop is stopped with an undamaged engine, except it to start turning again if the airspeed ever exceeds 95 KIAS.

\begin{figure}[htb]
%\includegraphics*[viewport = 0 25 385 265]{../Diagrams/CG_envelope2.ps} % original chart from Grace
  \begin{center}
  % GNUPLOT: LaTeX picture with Postscript
\begingroup
  \makeatletter
  \providecommand\color[2][]{%
    \GenericError{(gnuplot) \space\space\space\@spaces}{%
      Package color not loaded in conjunction with
      terminal option `colourtext'%
    }{See the gnuplot documentation for explanation.%
    }{Either use 'blacktext' in gnuplot or load the package
      color.sty in LaTeX.}%
    \renewcommand\color[2][]{}%
  }%
  \providecommand\includegraphics[2][]{%
    \GenericError{(gnuplot) \space\space\space\@spaces}{%
      Package graphicx or graphics not loaded%
    }{See the gnuplot documentation for explanation.%
    }{The gnuplot epslatex terminal needs graphicx.sty or graphics.sty.}%
    \renewcommand\includegraphics[2][]{}%
  }%
  \providecommand\rotatebox[2]{#2}%
  \@ifundefined{ifGPcolor}{%
    \newif\ifGPcolor
    \GPcolorfalse
  }{}%
  \@ifundefined{ifGPblacktext}{%
    \newif\ifGPblacktext
    \GPblacktexttrue
  }{}%
  % define a \g@addto@macro without @ in the name:
  \let\gplgaddtomacro\g@addto@macro
  % define empty templates for all commands taking text:
  \gdef\gplbacktext{}%
  \gdef\gplfronttext{}%
  \makeatother
  \ifGPblacktext
    % no textcolor at all
    \def\colorrgb#1{}%
    \def\colorgray#1{}%
  \else
    % gray or color?
    \ifGPcolor
      \def\colorrgb#1{\color[rgb]{#1}}%
      \def\colorgray#1{\color[gray]{#1}}%
      \expandafter\def\csname LTw\endcsname{\color{white}}%
      \expandafter\def\csname LTb\endcsname{\color{black}}%
      \expandafter\def\csname LTa\endcsname{\color{black}}%
      \expandafter\def\csname LT0\endcsname{\color[rgb]{1,0,0}}%
      \expandafter\def\csname LT1\endcsname{\color[rgb]{0,1,0}}%
      \expandafter\def\csname LT2\endcsname{\color[rgb]{0,0,1}}%
      \expandafter\def\csname LT3\endcsname{\color[rgb]{1,0,1}}%
      \expandafter\def\csname LT4\endcsname{\color[rgb]{0,1,1}}%
      \expandafter\def\csname LT5\endcsname{\color[rgb]{1,1,0}}%
      \expandafter\def\csname LT6\endcsname{\color[rgb]{0,0,0}}%
      \expandafter\def\csname LT7\endcsname{\color[rgb]{1,0.3,0}}%
      \expandafter\def\csname LT8\endcsname{\color[rgb]{0.5,0.5,0.5}}%
    \else
      % gray
      \def\colorrgb#1{\color{black}}%
      \def\colorgray#1{\color[gray]{#1}}%
      \expandafter\def\csname LTw\endcsname{\color{white}}%
      \expandafter\def\csname LTb\endcsname{\color{black}}%
      \expandafter\def\csname LTa\endcsname{\color{black}}%
      \expandafter\def\csname LT0\endcsname{\color{black}}%
      \expandafter\def\csname LT1\endcsname{\color{black}}%
      \expandafter\def\csname LT2\endcsname{\color{black}}%
      \expandafter\def\csname LT3\endcsname{\color{black}}%
      \expandafter\def\csname LT4\endcsname{\color{black}}%
      \expandafter\def\csname LT5\endcsname{\color{black}}%
      \expandafter\def\csname LT6\endcsname{\color{black}}%
      \expandafter\def\csname LT7\endcsname{\color{black}}%
      \expandafter\def\csname LT8\endcsname{\color{black}}%
    \fi
  \fi
  \setlength{\unitlength}{0.0500bp}%
  \begin{picture}(7200.00,5040.00)%
    \gplgaddtomacro\gplbacktext{%
      \csname LTb\endcsname%
      \put(1342,704){\makebox(0,0)[r]{\strut{}0}}%
      \csname LTb\endcsname%
      \put(1342,1247){\makebox(0,0)[r]{\strut{}2,000}}%
      \csname LTb\endcsname%
      \put(1342,1790){\makebox(0,0)[r]{\strut{}4,000}}%
      \csname LTb\endcsname%
      \put(1342,2332){\makebox(0,0)[r]{\strut{}6,000}}%
      \csname LTb\endcsname%
      \put(1342,2875){\makebox(0,0)[r]{\strut{}8,000}}%
      \csname LTb\endcsname%
      \put(1342,3418){\makebox(0,0)[r]{\strut{}10,000}}%
      \csname LTb\endcsname%
      \put(1342,3961){\makebox(0,0)[r]{\strut{}12,000}}%
      \csname LTb\endcsname%
      \put(1342,4504){\makebox(0,0)[r]{\strut{}14,000}}%
      \csname LTb\endcsname%
      \put(1474,484){\makebox(0,0){\strut{} 0}}%
      \csname LTb\endcsname%
      \put(2823,484){\makebox(0,0){\strut{} 5}}%
      \csname LTb\endcsname%
      \put(4172,484){\makebox(0,0){\strut{} 10}}%
      \csname LTb\endcsname%
      \put(5520,484){\makebox(0,0){\strut{} 15}}%
      \csname LTb\endcsname%
      \put(6869,484){\makebox(0,0){\strut{} 20}}%
      \put(308,2739){\rotatebox{-270}{\makebox(0,0){\strut{}Altitude (ft)}}}%
      \put(4171,154){\makebox(0,0){\strut{}Glide Distance (nm)}}%
      \put(1541,4639){\makebox(0,0)[l]{\strut{}Prop Windmilling}}%
      \put(1541,4368){\makebox(0,0)[l]{\strut{}Glide Ratio 8.5:1 (1.4 nm / 1000 ft)}}%
      \put(1541,4097){\makebox(0,0)[l]{\strut{}Flight test 16 Nov 2009}}%
    }%
    \gplgaddtomacro\gplfronttext{%
    }%
    \gplbacktext
    \put(0,0){\includegraphics{../graphs/glide}}%
    \gplfronttext
  \end{picture}%
\endgroup
\end{center}  % chart from gnuplot
  \caption{Engine-Out Glide Distance}
  \end{figure}


%\textcolor{red}{Figure 3-1 Maximum Glide}

If there is sufficient altitude to attempt to restart the engine, 
the procedure is to select Boost Pump ON, switch to the other
tank (provided it has fuel), select mixture to RICH and Alternate
Air ON. Check engine gauges for an indication of cause and if no fuel
pressure is indicated change tank selection. When power is restored
and at a safe altitude, select Alternate Air to OFF and turn Boost
Pump OFF.

If engine still fails to restart and time permits, turn each ignition
OFF, then ON to isolate a potentially bad ignition system. Try moving
the throttle and/or mixture to different settings. This may restore
power if mixture is too rich or too lean or if there is a partial
fuel blockage. Try the other tank, as water in the fuel may take time
to clear the system. Allowing the engine to windmill may restore power.
If failure is due to water then fuel pressure will be normal. Empty
fuel lines may take ten seconds to refill.

\subsubsection{ROUGH RUNNING ENGINE}

A slight engine roughness during flight may be caused by carbon or lead deposits fouling one or more spark plugs. 
This may be verified by selecting one ignition system OFF at a time.  A significant power loss in single ignition
operation is evidence of either spark plug fouling or an ignition system failure.  Assuming that fouled spark plugs is
the more likely cause, set cruise power and lean the engine to the recommended cruise fuel flows for several minutes. 
If the problem does not clear up after several minutes, determine if a richer mixture setting will result in smooth
running.

A sudden engine roughness or misfiring may be evidence of an ignition problem\ifthenelse{\thePMAG = 0}{(either magneto or electronic ignition)
 }{}. Switching one ignition system off in turn will identify which one is malfunctioning.  Select different power settings
and enrichen the mixture to determine whether continued operation on both ignition systems is possible.

If the problem continues, try different mixture and throttle settings. Select Boost Pump ON,
change fuel tanks and select Alternate Air ON. Select each ignition system
OFF then ON.

\begin{Note}[CAUTION]
The engine may quit completely when one ignition is selected OFF, if the other ignition is faulty.  
If this occurs, to prevent a severe afterfire (unburnt fuel exploding in exhaust system), select the mixture to ICO, wait 10 seconds, then select the ignition back ON.
Advance the mixture slowly until engine restarts.
\end{Note}

%\subsubsection{LOW FUEL PRESSURE}
%
%If fuel pressure falls, select Boost Pump ON and change fuel tanks.

\subsubsection{LOW OIL PRESSURE}

If low oil pressure is accompanied by normal oil temperature, there is a possibility that there is a leak in the line
from the engine to the oil pressure transducer manifold, or that the oil pressure relief valve has malfunctioned.  The
EIS 4000
will flash the ENGINE WARN light if the oil pressure decreases below limits. The LOW OIL PRESS light is driven by a
different transducer on the same external manifold. If only one system indicates low oil pressure it is almost certainly
a false indication.

A leak in the line to oil pressure transducer manifold is not
necessarily cause for immediate precautionary landing, as an orifice in the line will prevent sudden loss of oil from
the sump.  However the aircraft should be landed at the nearest airport for inspection.

A total loss of oil pressure accompanied by increasing oil temperature may indicate an impending total engine failure.
An off airfield landing while power is available is strongly recommended, especially in the presence of additional
indicators such as a rise in engine CHT or oil temperature, oil and/or smoke apparent. 

\subsubsection{HIGH OIL TEMPERATURE}

High oil temperature may be caused by the oil cooler door being closed
too far. High oil temperature may also be due to a low oil level,
obstruction in oil cooler (internal or external), damaged baffle seals,
a defective gauge, or other causes. The EIS 4000 ENGINE WARN light
will flash if the oil temperature becomes higher than 225\textdegree F. A steady
rise in oil temperature is a particular sign of trouble.

Always land as soon as possible at an appropriate airport/airfield
and investigate and be prepared for an engine failure. Open the oil
cooler door. Keep the airspeed up to maximize airflow over the engine.
Watch the oil pressure and CHT (Cylinder Head Temperature) gauge to
identify impending failure.

\subsection{FORCED LANDINGS}
\subsubsection{POWER OFF LANDING}

The initial action is ALWAYS TRIM FOR BEST GLIDE, 95 to 115 KIAS, depending
on weight. If engine power is not restored and time allows check for
airports/strips available and notify of problem/intent if possible.  
Select Mixture to IDLE CUT OFF.  
Closing the throttle and pulling the prop control full aft will significantly reduce windmilling drag if oil pressure
is available.  Select Fuel Selector to OFF and Ignition Switches to OFF. 
Transmit a MAYDAY.

Identify a suitable field, planning an into wind landing. Try to be
1000 ft AGL at the end of the downwind leg to make a normal landing.
Aim initially for the centre of the field (drag with a wind milling
propeller may be higher than expected) and only lower final stages
of flap when there is no doubt the field can be reached. Sideslip
as required to lose excess altitude. Plan for slowest short field
landing but above all else do not stall.

When committed to landing extend Flaps to FULL then select BATT/ALT switch to OFF. 
 Seat belts should be tight and touchdown made at the
slowest speed possible.

\textcolor{red}{Add Power-On Landing and Ditching}
\subsection{FIRES}
%\subsubsection{ENGINE FIRE DURING START}
%
%\textcolor{red}{Nothing to add to checklist?} 
%

\subsubsection{ENGINE FIRE IN FLIGHT}

The key to dealing with an engine fire is to stop the flow of fuel
to the engine compartment. Put the mixture to IDLE CUT-OFF, switch
Fuel Selector OFF, and select Boost Pump OFF. Close cabin heat and
air vents. All electrical power can be removed ahead of the firewall
by selecting the ESS BUS FEED switch to EMER (to keep the avionics powered to allow a Mayday call) and the BATT/ALT switch to OFF.

\subsubsection{ELECTRICAL FIRE}

The EFIS internal battery will power it for 3 hours (if it is fully charged), so the aircraft's entire electrical system may be 
shutdown if required due to electrical smoke or fire.  The EFIS will automatically shutdown 30 seconds
after the electrical power is cut, but the automatic shutdown sequence may be cancelled by a momentary
press of the left-most EFIS button.  The Turn Coordinator is powered from the Battery Bus, 
and will continue to be powered even if the whole electrical system is shutdown.

Fight the fire with the fire extinguisher, if required.  
Close the vents prior to using the fire extinguisher, to increase its effectiveness.
Open the vents after using the fire extinguisher to ventilate the cockpit.

%\subsubsection{CABIN FIRE}
%
%Shutdown the electrical system.  Fight the fire with the fire extinguisher, if required.  
%Close the vents prior to using the fire extinguisher, to increase its effectiveness.
%Open the vents after using the fire extinguisher to ventilate the cockpit.
%
%\subsubsection{WING FIRE}
%
%Remove electrical power to items in the wing.  Select Nav/Strobe, Landing and Taxi Lights and Pitot Heat OFF.

\subsection{OTHER}
\subsubsection{STATIC SOURCE BLOCKAGE}

If erroneous readings of the static system instruments (airspeed, altimeter and rate of climb) are suspected, the
alternate static source valve should be opened, thus supplying air to those instruments from inside the cockpit.

To avoid the possibility of large errors, the fresh air vents and cockpit heat should be closed. 
Adjust the altitude and airspeed readings by the corrections shown in Section
\ref{perf-sec-number}.

\subsubsection{INADVERTENT ICING ENCOUNTER}
\begin{enumerate*}
  \item Confirm pitot heat is selected ON.
  \item Turn back or change altitude to obtain an outside air temperature that is less conducive to icing.
  \item Increase RPM to minimize ice build-up on the propeller blades.
  \item Plan a landing at the nearest airport.  With extremely rapid ice build-up, select a suitable ``off-airport'' landing site.
  \item Minimize flap extension if there is an ice build-up on the horizontal stabilizer, to reduce the risk of tail plane stall.
  \item Use sideslip as required to improve forward visibility if the windscreen is obscured by ice.
  \item Increase the approach speed by at least 10 kt, to reduce the risk of wing stall.
  \end{enumerate*} 

\subsubsection{ALTERNATOR FAILURE}

Alternator failure is identified by a low voltage indication from
the EIS 4000, which will cause the ENG WARN light to flash.

The root fault may have been an over-voltage condition, in which case
the over-voltage protection system will cause the ALT CB to trip,
leading to a low voltage condition. While it is possible to put the
alternator back on line by resetting the ALT CB, this is not recommended
unless necessary for safety of flight.

The BATT/ALT switch should be selected to BATT, then BATT + ALT to attempt
to reset the alternator. If the alternator cannot be brought back
on line the Standby Alternator can be used to provide up to 8 amps
of power. Select the ESS BUS FEED to EMER, which provides power
to the Essential Bus directly from the Battery Bus, bypassing the
Battery Contactor. The Main Bus can then be shed by selecting the
BATT/ALT switch to OFF. The EFIS will continue to run using its internal
battery, however it can be restored to aircraft power by selecting
the EFIS BU switch to ON. The loads on the Essential Bus and
Battery Bus total less than 8 amps.

The following systems are unpowered in this configuration:

\begin{itemize*}
\item Boost Pump
\item Pitot Heat 
\item Landing Light 
\item Taxi Light 
\item Position Lights 
\item Strobe Lights 
\item Engine Instrument Lighting (Map light on goose neck may be used to
illuminate instrument panel) 
\item CDI Lighting 
\item Flaps 
\item Starter 
\item Wing Leveler 
\item Microair 760 Com (COM 2) 
\item Intercom 
\item Low Oil Pressure Warning Light 
\item Analog Tach and MP
\item Defrost Fan
\end{itemize*}
The BATT/ALT switch may be selected to ON prior to landing to restore
power to all systems.

\subsubsection{STARTER ENGAGED LIGHT ILLUMINATED}
The STARTER ENGAGED caution light illuminates whenever the starter is engaged.  If this light remains illuminated after
the starter switch has been released, it indicates that the starter relay is stuck closed, and the starter is still
powered.
 Continuing to supply power to the
starter will eventually result in complete loss of electrical system power, substantial starter damage, and possible
damage to other electrical system components.  If the starter is stuck closed, the only way to remove power from the
starter is to select the BATT/ALT switch to OFF.


\subsubsection{WING-LEVELER MALFUNCTION}

Press the red Wing-Leveler/Trim Disconnect button on the control stick, and hold it.  This removes power from the servo, and should cause it to release.
Select the Wing-Leveler Power switch OFF (RH console), then release the Wing-Leveler/Trim Disconnect button.  
%If required, the control rod from the servo to the controls may be disconnected by pulling the ``pip'' pin located at the base of the front control stick, between the stick and the front spar area.

\subsubsection{RUNAWAY TRIM}

Press the red Wing-Leveler/Trim Disconnect button on the control stick, and hold it.  Select the Trim Power switch OFF (RH console), then release the Wing-Leveler/Trim Disconnect button.

\subsubsection{HIGH CO}

Decrease power.  Set mixture well lean of peak (stops CO production).  Close both cabin heat controls.  Open the vents.  Monitor the CO Monitor readings.  
Immediate action is required if the red light is illuminated.  
Consider slowing the aircraft and cracking the canopy open if there is not an immediate decrease in the CO levels.

%\subsubsection{LANDING WITH NO ELEVATOR CONTROL}
%\textcolor{red}{Add this item after in-flight assessment}

\subsubsection{SPIN RECOVERY}

Upright spin - The upright spin recovery procedure is:
\begin{enumerate*}
  \item Retard throttle to idle.
  \item Retract flaps, if extended.
  \item Place ailerons to neutral position.
  \item Apply and hold full rudder opposite to spin direction.  If the visibility of the ground does not allow the spin direction to be determined, use the turn needle --- depress the rudder on the opposite side to that of the turn needle.
  \item Just after the rudder reaches the stop, move the stick briskly forward far enough to break the stall.
  \item Hold these control inputs until rotation stops.  Premature relaxation of the recovery control inputs may prevent spin recovery.
  \item As rotation stops, neutralize the rudder, and recover from the dive, using g as required to stay below $V_{NE}$ without exceeding the load factor limit.  Aggressive use of g early in the recovery will minimize the airspeed build-up.
  \end{enumerate*}
  
\begin{Note}
The inverted spin recovery procedure has not been validated via flight test.
\end{Note}

Inverted spin - The inverted spin recovery procedure is:
\begin{enumerate*}
  \item Retard throttle to idle.
  \item Retract flaps, if extended.
  \item Place ailerons to neutral position.
  \item Confirm spin direction by reference to the turn needle --- the turn needle moves to the side of the spin direction.  Apply and hold full rudder opposite to spin direction.  
  \item Just after the rudder reaches the stop, move the stick briskly \textbf{aft} far enough to break the stall.
  \item Hold these control inputs until rotation stops.  Premature relaxation of the recovery control inputs may prevent spin recovery.
  \item As rotation stops, neutralize the rudder, and recover from the dive, using g as required to stay below $V_{NE}$ without exceeding the load factor limit.  Aggressive use of g early in the recovery will minimize the airspeed build-up.
  \end{enumerate*}
