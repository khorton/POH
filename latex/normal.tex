% !iTeXMac(input): POH.tex
\chapter{NORMAL PROCEDURES}
\vspace{\minitocspacebefore}
\minitoc
\cleardoublepage

\section{GENERAL}

Pilots should familiarize themselves with the procedures in this section
to become proficient with the normal safe operation of the aircraft.

\section{AIRSPEEDS FOR NORMAL OPERATION}
%(\textcolor{red}{Red text indicates provisional information, based
%on the RV-8A Aircraft Performance Report published by the CAFE Foundation}.)

\begin{center}\begin{tabular}{llr}
$\mathrm{V_{R}}$&
Takeoff rotate speed&
60\tabularnewline
%&
%Normal Takeoff, speed at 50 ft &
%\textcolor{red}{70}\tabularnewline
%&
%Short Field Takeoff, speed at 50 ft&
%\textcolor{red}{TDB}\tabularnewline
$\mathrm{V_{Y}}$ &
Best rate of climb speed, Sea Level&
102 KIAS\tabularnewline
$\mathrm{V_{X}}$ &
Best angle of climb speed, Sea Level&
65 KIAS\tabularnewline
$\mathrm{V_{BG}}$&
Best glide angle&
115 KIAS\tabularnewline
$\mathrm{V_{A}}$ &
Manoeuvring speed&
120 KIAS\tabularnewline
$\mathrm{V_{S_{0}}}$ &
Stall Full Flap&
51 KIAS\tabularnewline
$\mathrm{V_{S}}$ &
Stall Flaps UP&
55 KIAS\tabularnewline
$\mathrm{V_{FE}}$ &
Maximum speed with flaps extended &
87 KIAS\tabularnewline
$\mathrm{V_{REF}}$ &
Final approach speed (full flap)&
65\tabularnewline
$\mathrm{V_{REF0}}$ &
Final approach speed (zero flap)&
70\tabularnewline
&
Demonstrated crosswind velocity&
25\tabularnewline
\end{tabular}\end{center}

%\addpenalty{-10000}
%\section{ENGINE OPERATING CONDITIONS}
%
%\begin{center}\begin{tabular}{lccccc}
%&
%RPM&
%HP&
%GPH&
%Max Oil Consumption&
%Max. CHT\tabularnewline
%&
%&
%&
%&
%(Qts/Hr)&
%(deg C)\tabularnewline
%Normal Rated&
%2700&
%200&
%--&
%.89&
%260\tabularnewline
%Performance Cruise (75\%)&
%2450&
%150&
%12.3&
%.5&
%260\tabularnewline
%Economy Cruise (65\%)&
%2350&
%130&
%9.5&
%.44&
%260\tabularnewline
%\end{tabular}\end{center}

\cleardoublepage
% !iTeXMac(input): POH.tex
%\section*{NORMAL CHECKLISTS} % leave this line commented out, 
%as it destroys the visual flow of the checklist page.
% Find a way to get "NORMAL CHECKLIST" in the header instead.
%\addcontentsline{toc}{section}{\numberline {NCL}Normal Checklist}
\changepage{\checklistTextHeight}{\checklistTextWidth}{\checklistEvenSideMargin}{\checklistOddSideMargin}{}{\checklistTopMargin}{0pt}{0pt}{0pt}% original settings, with no room for header or footer

%\addcontentsline{toc}{section}{\numberline { }NORMAL CHECKLIST}
\addcontentsline{toc}{section}{NORMAL CHECKLIST}
%\ohead{NORMAL CHECKLIST}% puts "NORMAL CHECKLIST" in the header.
\chead{\Large NORMAL CHECKLIST} % puts "NORMAL CHECKLIST" in the header.
% \ohead{DRAFT --- \thistime \ \cdndate\today}
\ohead{Created --- \thistime \ \cdndate\today}

\begin{multicols}{2}
% \raggedcolumns

\subsection*{PREFLIGHT INSPECTION }

\subsubsection*{COCKPIT}
\begin{enumerate*}
\item Ignition Switches\dotfill OFF
\item BATT/ALT \dotfill BATT
\item Fuel Gauges \dotfill CHECK
\item Fuel Selector \dotfill LOWEST TANK
\item Landing \& Taxi Lights \dotfill CHECK A/R
\item Flaps \dotfill EXTEND
\item BATT/ALT \dotfill OFF
\item Alternate Static Source Hose \dotfill SECURE
\item Rear Stick \dotfill A/R 
\item Rear Seat Belts \dotfill A/R
\item Rear Cockpit Side Pockets \dotfill A/R
\item ELT Antenna \dotfill SECURE
\item Rear Baggage Area \dotfill SECURE
\item ELT \dotfill ARM
\end{enumerate*}

\subsubsection*{LEFT WING }
\begin{enumerate*}
\item Flap Pushrod End \dotfill SECURE
\item Aileron Pushrod \dotfill SECURE
\item Aileron Hinge \dotfill SECURE
\item Nav Light Cover \dotfill SECURE
\item Taxi Light \dotfill SECURE
\item Pitot Tube \dotfill CLEAR
\item Fuel Quantity \dotfill VISUAL CHECK
\item Fuel Tank Drain \dotfill $\mathrm{H_{2}O}$ CHECK
\item Gascolator Drain \dotfill $\mathrm{H_{2}O}$ CHECK
\end{enumerate*}

\subsubsection*{FORWARD FUSELAGE }
\begin{enumerate*}
\item L Tire \& Wheel Pant \dotfill CHECK COM
\item Antenna \dotfill SECURE
\item Fuel vents \dotfill CLEAR
\item Exhaust Pipes \dotfill SHAKE ENDS
\item Cowl \dotfill SECURE
\item Air Inlets \dotfill CLEAR
\item Air Filter \dotfill CHECK
\item Spinner \dotfill SECURE
\item Prop \dotfill CHECK
\item Oil Quantity \dotfill CHECK 
\item Oil Door \dotfill SECURE
\item Fwd Baggage \dotfill ITEMS CHECK
\item Windshield \dotfill CLEAN
\item Transponder Antenna \dotfill SECURE
\item R Tire \& Wheel Pant \dotfill CHECK
\end{enumerate*}

\subsubsection*{RIGHT WING }
\begin{enumerate*}
\item Fuel Tank Drain \dotfill $\mathrm{H_{2}O}$ CHECK 
\item Fuel Quantity \dotfill VISUAL CHECK
\item Landing Light \dotfill SECURE
\item Nav Light Cover \dotfill SECURE
\item Aileron Hinge \dotfill SECURE
\item Aileron Pushrod \dotfill SECURE
\item Flap Pushrod End \dotfill SECURE
\end{enumerate*}

\subsubsection*{RIGHT REAR FUSELAGE }
\begin{enumerate*}
\item GPS Antenna \dotfill SECURE
\item Canopy \dotfill CHECK
\item Static Port \dotfill CLEAR
\end{enumerate*}

\subsubsection*{EMPENNAGE}
\begin{enumerate*}
\item Empennage Fairing Top \dotfill SECURE
\item Empennage Fairing Bottom \dotfill SECURE
%\item Empennage Fairing \dotfill Secure round elevator
\item Empennage Fairing \dotfill SECURE AROUND ELEV.
\end{enumerate*}

\subsubsection*{LEFT REAR FUSELAGE}
\begin{enumerate*}
\item Static Port \dotfill CLEAR
\item Canopy \dotfill CHECK
\end{enumerate*}
\end{multicols} % end multicols so that the "This page intentionally left blank" will go in center of page.
\cleardoublepage

\begin{multicols}{2}
\raggedcolumns
\subsection*{IN-FLIGHT CHECKLISTS}

\subsubsection*{BEFORE START}
\begin{enumerate*}
\item Seat Belts \dotfill SECURE
\item Controls \dotfill FREE \& CORRECT
\item Throttle \dotfill FULL OPEN
\item Prop \dotfill LO RPM
\item Mixture \dotfill ICO
\item Alternate Air \dotfill CLOSED
\item Oil Cooler Door \dotfill A/R
% \item CO Monitor \dotfill OFF
\item Ignition Switches \dotfill OFF
% \item Inst Panel Switches \dotfill OFF
\item All Switches and Avionics \dotfill OFF or NORM
% \item Avionics \dotfill OFF
% \item RH Console Switches \dotfill OFF or NORM
\item ESS BUS FEED \dotfill EMER
\item ENG INST \dotfill ON
\item TRIM \dotfill FRONT or FRONT + REAR
\item Trim \dotfill CHECK
\item Trim Cut-out \dotfill CHECK
\item EIS 4000 \dotfill ENTER FUEL QTY
%\item COM 2 \dotfill ON
%\item ATIS/Clearance \dotfill OBTAIN
%\item Altimeter \dotfill SET
\item BATT/ALT \dotfill BATT + ALT
\item ESS BUS FEED \dotfill NORM
\item Engine Warn Light \dotfill CHECK ON
\item Oil Press Light \dotfill CHECK ON
\item Nav Lights \dotfill A/R
% \item Intercom \dotfill CHECK
\end{enumerate*}

\subsubsection*{ENGINE START (Cold /Warm)}
\begin{enumerate*}
% \item Seat Belts \dotfill SECURE
% \item Controls \dotfill FREE \& CORRECT
% \item Canopy \dotfill A/R
\item Fuel Selector \dotfill LEFT or RIGHT
% \item Strobe\dotfill A/R
\item IGNITION, MAG \dotfill ON
\item Starter Switch \dotfill ENABLE
%\item Throttle \dotfill FULL OPEN
%\item Mixture \dotfill ICO
\item Boost Pump \dotfill ON
\item Mixture \raggedright \dotfill FULL RICH 4-5 SEC - COLD\\\dotfill FULL RICH 2-3 SEC WARM
% \item Mixture \raggedright \dotfill FULL RICH 5 SEC - COLD\\\hfill FULL RICH 2-3 SEC - WARM
\item Mixture \dotfill ICO
\item Boost Pump \dotfill OFF
\item Throttle \dotfill 1/2" OPEN
\item Call \dotfill CLEAR
\item Starter \dotfill ENGAGE
\item IGNITION, ELEC \dotfill ON (after two blades)
\suspend{enumerate*}
\begin{Note}[CAUTION]
Do not select Electronic Ignition ON until the engine
has turned two blades. Failure to observe this restriction may result
in engine kick-back and starter damage.
%Do not select EI ON prior to starter engagement. The EI may
%be selected ON while the starter is engaged, as long as the engine
%has turned two blades. Failure to observe this restriction may result
%in engine kick-back and starter damage.
\end{Note}
\resume{enumerate*}
\item Mixture \dotfill RICH WHEN ENG FIRES
\item Engine \dotfill 1000 RPM
\item Oil Pressure \dotfill CHECK (30 s)
\end{enumerate*}

\subsubsection*{ENGINE START (Hot)}
\begin{enumerate*}
% \item Seat Belts \dotfill SECURE
% \item Controls \dotfill FREE \& CORRECT
% \item Canopy \dotfill A/R
\item Fuel Selector \dotfill LEFT or RIGHT
% \item Strobe \dotfill ON
\item IGNITION, MAG \dotfill ON
\item Starter Switch \dotfill ENABLE
%\item Throttle \dotfill FULL OPEN
\item Mixture \dotfill ICO
\item Boost Pump \dotfill ON
\item Call \dotfill CLEAR
\item Starter \dotfill ENGAGE
\item IGNITION, ELEC \dotfill ON (after two blades)
\suspend{enumerate*}
\begin{Note}[CAUTION]
Do not select Electronic Ignition ON until the engine
has turned two blades. Failure to observe this restriction may result
in engine kick-back and starter damage.
%Do not select EI ON prior to starter engagement. The EI may
%be selected ON while the starter is engaged, as long as the engine
%has turned two blades. Failure to observe this restriction may result
%in engine kick-back and starter damage.
\end{Note}
\resume{enumerate*}
\item Mixture \dotfill RICH WHEN ENG FIRES
\item Throttle \dotfill RETARD
\item Engine \dotfill 1000 RPM
\item Oil Pressure \dotfill CHECK (30 s)
\item Boost Pump \dotfill OFF
\end{enumerate*}

\subsubsection*{AFTER START}
\begin{enumerate*}
\item OIL PRESS Light\dotfill OUT
\item STARTER ENGAGED Light \dotfill OUT
\item STARTER Switch\dotfill OFF \& Guarded
\item FLAPS \dotfill UP
\item Voltage \dotfill CHECK
\item Fuel Pressure \dotfill CHECK
%\item GNS 430 \dotfill ON
%\item Transponder \dotfill STBY, CODE SET
%\item COM 2 \dotfill ON
%\item Narco 122D \dotfill ON
\item Avionics \dotfill ON
\item TURN COORD \dotfill ON
\item WING LVLR \dotfill ON
\item CO Monitor Self Test \dotfill CHECK
\item ATIS/Clearance \dotfill OBTAIN
\item Altimeter \dotfill SET
%\item GNS 430 \dotfill ENTER FPLAN
\end{enumerate*}

\subsubsection*{TAXI}
\begin{enumerate*}
% \item TAXI LT \dotfill ON
\item Brakes \dotfill CHECK
\item Flight Instruments \dotfill CHECK
\item \emph{For IFR or Night Operations ...}
\begin{itemize*}
\item Autopilot Override \dotfill CHECK
\end{itemize*}
\end{enumerate*}

\subsubsection*{RUNUP}
\begin{enumerate*}
\item Fuel Selector \dotfill CHANGE TANKS
\item Mixture \dotfill RICH
\item Throttle \dotfill 1800 RPM
\item Prop \dotfill CYCLE
\item IGNITION, ELEC \dotfill OFF/ON
\item IGNITION, MAG \dotfill OFF/ON
\item \emph{For IFR or Night Operations ...}
\begin{itemize*}
\item ESS BUS FEED \dotfill EMER
\item BATT/ALT \dotfill OFF
\item STBY ALT \dotfill ON
\item Voltage \dotfill CHECK
\item STBY ALT \dotfill OFF
\item BATT/ALT \dotfill BATT + ALT
\item ESS BUS FEED \dotfill NORM
\end{itemize*}
\item Voltage \dotfill CHECK
\item Throttle \dotfill IDLE CHECK
\item Mixture \dotfill LEAN
\end{enumerate*}

\columnbreak % added to force all of runup checklist onto one page

\subsubsection*{BEFORE TAKEOFF}
\begin{enumerate*}
%\item SPOT \dotfill TRACK MODE
\item Seat Belts \dotfill SECURE
\item Flight Controls \dotfill FREE
\item Trims \dotfill SET
\item Flaps \dotfill SET
\item Prop \dotfill FULL FWD
\item Ignition Switches\dotfill BOTH ON
\item Alternate Air \dotfill CLOSED
\item Oil Cooler Door \dotfill A/R
\item Radios/Navaids \dotfill SET
\item Transponder \dotfill CODE SET + ALT
%\item Wing Leveler \dotfill TC
\item Altimeter \dotfill SET/CHECK
\item Engine Instruments \dotfill CHECK
\item MSTR WARN Light \dotfill OUT
% \item Fuel \dotfill CHECK
\item Fuel Selector \dotfill FULLEST TANK
\item T/O Brief \dotfill COMPLETE\vspace{0.5ex}
\hrule width \columnwidth \vspace{1ex}
\item Canopy \dotfill LATCHED
\item PITOT HEAT \dotfill A/R
\item LDG LT \& TAXI LT \dotfill ON or FLASH
\item NAV/STR \dotfill NAV + STR
\item Mixture \dotfill RICH
\item Boost Pump \dotfill ON
\item Compasses \dotfill CHECK
\end{enumerate*}

\subsubsection*{AFTER TAKEOFF}
\begin{enumerate*}
\item Flaps \dotfill UP
\item LDG LT \& TAXI LT \dotfill A/R
% \item Transponder \dotfill CONFIRM ALT
\item Boost Pump \dotfill OFF
\item Power \dotfill 2500 rpm/Full Throttle
\end{enumerate*}

\subsubsection*{CRUISE}
\begin{enumerate*}
\item Power \dotfill SET
\item Mixture \dotfill SET
\item Fuel \dotfill CHECK
\item Oil Cooler Door \dotfill A/R
% \item Fuel Selector \dotfill CHANGE TANKS EVERY 30 MN
\end{enumerate*}

\columnbreak
\subsubsection*{AEROBATICS}
\begin{enumerate*}
\item Fuel \dotfill LEFT TANK
\item Mixture \dotfill RICH
\item Harness \dotfill TIGHT
\item Loose Items \dotfill STOW
\item Area \dotfill CLEAR
\end{enumerate*}

\subsubsection*{DESCENT}
\begin{enumerate*}
\item Parking Brake \dotfill OFF
\item Altimeter \dotfill SET
\item Approach \dotfill BRIEF
\item Oil Cooler Door \dotfill A/R
\item LDG LT \& TAXI LT \dotfill A/R
\end{enumerate*}

\subsubsection*{BEFORE LANDING}
\begin{enumerate*}
\item Seat Belts \dotfill SECURE
\item Fuel Selector \dotfill FULLEST TANK
\item Mixture \dotfill RICH
\item Boost Pump \dotfill ON
% \item Wing Leveler \dotfill TC
\item Prop \dotfill FULL FWD
\item Flaps \dotfill A/R
% \item LDG LT \& TAXI LT \dotfill A/R
% \item Airspeed \dotfill XX KT
\end{enumerate*}

\subsubsection*{AFTER LANDING}
\begin{enumerate*}
\item PITOT HEAT \dotfill OFF
\item Mixture \dotfill LEAN
\item Oil Cooler Door \dotfill OPEN
\item Boost Pump \dotfill OFF
% \item Flaps \dotfill UP
% \item NAV/STR \dotfill OFF or NAV
% \item TAXI LT \dotfill A/R
% \item LDG LT \dotfill OFF
\item External Lights \dotfill A/R
\item Transponder \dotfill STBY
\item Flaps \dotfill 90\% DOWN
\end{enumerate*}

\subsubsection*{SHUTDOWN}
\begin{enumerate*}
\item Avionics \dotfill OFF
% \item ENG INST \dotfill OFF
% \item TURN COORD \dotfill OFF
% \item DFRST FAN \dotfill OFF
% \item TRIM \dotfill OFF
% \item WING LVLR \dotfill OFF
\item Throttle \dotfill IDLE
\item Dead Mag \dotfill CHECK
\item Mixture \dotfill ICO
\item Ignition Switches \dotfill BOTH OFF
\item Fuel Selector \dotfill OFF
% \item NAV/STR \dotfill OFF
\item All Switches \dotfill OFF or NORM
%\item SPOT \dotfill OFF
%\item BATT/ALT \dotfill OFF
\end{enumerate*}
\end{multicols}

\cleardoublepage
\ohead{\leftmark} % put Section name back in outer header
\chead{} % remove center header
\changepage{\checklistTextHeight*-1}{\checklistTextWidth*-1}{\checklistEvenSideMargin*-1}{\checklistOddSideMargin*-1}{}{\checklistTopMargin*-1}{0pt}{0pt}{0pt}% original settings, with no room for header or footer



\cleardoublepage
\changepage{\checklistTextHeight}{\checklistTextWidth}{\checklistEvenSideMargin}{\checklistOddSideMargin}{}{\checklistTopMargin}{0pt}{0pt}{0pt}% original settings, with no room for header or footer

%\addcontentsline{toc}{section}{\numberline { }NORMAL CHECKLIST}
\addcontentsline{toc}{section}{CHEAT SHEET}
%\ohead{NORMAL CHECKLIST}% puts "NORMAL CHECKLIST" in the header.
\chead{\Large CHEAT SHEET} % puts "NORMAL CHECKLIST" in the header.
% \ohead{DRAFT --- \thistime \ \cdndate\today}
\ohead{Created --- \thistime \ \cdndate\today}

\begin{multicols}{2}
\raggedcolumns

\subsection*{EIS 4000}
\subsubsection*{CHANGE FUEL QUANTITY TO FULL FUEL}
\begin{enumerate*}
\item L and R Buttons\dotfill PRESS AND HOLD
\item INC and DEC Soft Keys\dotfill PRESS (to set 42 USG)
\item NEXT Soft Key\dotfill PRESS (to return to normal display)
\end{enumerate*}

\subsubsection*{CHANGE FUEL QUANTITY TO ANY VALUE}
\begin{enumerate*}
\item L and R Buttons\dotfill PRESS AND HOLD
\item INC or DEC Soft Keys\dotfill PRESS (to set desired quantity)
\item NEXT Soft Key\dotfill PRESS (to return to normal display)
\end{enumerate*}

\subsubsection*{SET ALARM LIMITS}
\begin{enumerate*}
\item L and C Buttons\dotfill PRESS AND HOLD FOR 10S
\item UP or DOWN Soft Keys\dotfill PRESS (to modify parametres)
\item NEXT Soft Key\dotfill PRESS (to advance to next parametre)
\item NEXT Soft Key\dotfill PRESS REPEATEDLY (to return to normal display)
\end{enumerate*}

\subsubsection*{ACCESS CONFIGURATION MENUS}
\begin{enumerate*}
\item C and R Buttons\dotfill PRESS AND HOLD FOR 10S
\item UP or DOWN Soft Keys\dotfill PRESS (to modify parametres)
\item NEXT Soft Key\dotfill PRESS (to advance to next parametre)
\item EIS Power\dotfill OFF-ON (to return to normal display)
\end{enumerate*}

\subsubsection*{LEANING MODE}
\begin{enumerate*}
\item Centre and R Buttons\dotfill PRESS AND HOLD
\end{enumerate*}


\columnbreak % added to force all of GNS 430W stuff onto one page
\subsection*{GNS 430W}
\subsubsection*{REVIEW FLT PLAN LATS \& LONGS}
\begin{enumerate*}
\item FPL Page\dotfill SELECT
\item Cursor\dotfill ACTIVATE
\item Waypoint Name\dotfill SELECT
\item ENT Button\dotfill PUSH (to see wpt info)
\item Lat/Long\dotfill REVIEW
\item ENT Button\dotfill PUSH (to move to next wpt)
\end{enumerate*}

\subsection*{SPOT}
\subsubsection*{START TRACKING}
\begin{enumerate*}
\item Power Button\dotfill PRESS AND HOLD (until lights flash)
\item Footprint Button\dotfill PRESS AND HOLD (until light flashes)
\end{enumerate*}

\subsubsection*{SEND CHECK-IN MESSAGE}
\begin{enumerate*}
\item Power Button\dotfill PRESS AND HOLD (until lights flash)
\item Checkmark/OK Button\dotfill PRESS AND HOLD (until light flashes)
\end{enumerate*}

\subsubsection*{SEND CUSTOM MESSAGE}
\begin{enumerate*}
\item Power Button\dotfill PRESS AND HOLD (until lights flash)
\item Text Bubble Button\dotfill PRESS AND HOLD (until light flashes)
\end{enumerate*}

\subsubsection*{SEND REQUEST FOR ASSISTANCE (non life threatening)}
\begin{enumerate*}
\item Power Button\dotfill PRESS AND HOLD (until lights flash)
\item Holding Hands Button\dotfill PRESS AND HOLD (until light flashes)
\end{enumerate*}

\subsubsection*{SEND EMERGENCY REQUEST FOR ASSISTANCE}
\begin{enumerate*}
\item Power Button\dotfill PRESS AND HOLD (until lights flash)
\item SOS Button\dotfill LIFT TAB
\item SOS Button\dotfill PRESS AND HOLD (until light flashes)
\end{enumerate*}

\subsubsection*{CANCEL EMERGENCY REQUEST FOR ASSISTANCE}
\begin{enumerate*}
\item Power Button\dotfill PRESS AND HOLD (until lights flash)
\item SOS Button\dotfill LIFT TAB
\item SOS Button\dotfill PRESS AND HOLD (until light flashes red)
\item SPOT\dotfill LEAVE POWERED (until SOS button stops flashing red)
\end{enumerate*}

\end{multicols}

\cleardoublepage
\ohead{\leftmark} % put Section name back in outer header
\chead{} % remove center header
\changepage{\checklistTextHeight*-1}{\checklistTextWidth*-1}{\checklistEvenSideMargin*-1}{\checklistOddSideMargin*-1}{}{\checklistTopMargin*-1}{0pt}{0pt}{0pt}% original settings, with no room for header or footer


\section{AMPLIFIED PROCEDURES}

\subsection{PREFLIGHT INSPECTION}
The following items should receive particular emphasis during the preflight inspection.

\textbf{Propeller} --- Check the amount of free play in the propeller blades, moving the blade tips in the plane of rotation. 1/8\char`\"{} of free play is acceptable. 2 degrees of blade rotation angle free play is acceptable. Check that the stainless steel erosion strips on the outer portion of the blade leading edges is secure. The inner PU-strip self-adhesive shield may be missing, but should be replaced within 10 hours.

Check the fibreglass blade covers for cracks. Cracks along the leading edge and at the edge of the erosion shield are acceptable, as long as the erosion shield is not loose. Cracks in the painted surface are acceptable, as long as no moisture can enter the wooden blade core. Blisters or delaminations up to one square inch are acceptable. Cracks in the stainless steel erosion shield require immediate repair.

\textbf{Fuel Tank Vents} --- The fuel vent plugs are easily missed, as the fuel vents are below the forward fuselage. Ensure that they are removed.

\textbf{Flaps}
\begin{Note}[CAUTION]
With the flaps fully extended, there is a risk that someone standing behind the wing could press on flap trailing edge, and force the flaps to move far enough that the forward edge of the upper flap skin could move aft of the aft edge of the upper wing skin. If this condition is not detected significant flap and/or upper wing skin damage may occur when the flaps are retracted. During the preflight inspection, carefully inspect the transition between the upper wing skin and the flaps to confirm that the forward edge of the upper flap skin has not come out from under the aft edge of the upper wing skin. When boarding the aircraft after the preflight inspection, be careful not to push forward on the trailing edge of the flap. 

The risk of forcing the flaps far enough to trigger this issue is mitigated by slightly retracting the flaps after landing.
\end{Note}


\textbf{Tires} --- The tires are almost completely hidden beneath the wheel pants, so it can be difficult to see if the tire pressure is low. Carefully check the clearance between the tire sidewalls and the wheel pant openings.

\textbf{Pitot-Static System} --- Ensure that the pitot cover is removed from the pitot tube, which is hidden beneath the left wing. The static system hose connected to the alternate static valve is easily knocked loose (mounted on the side of the right landing gear box in the cockpit). Check it for security, and be careful not to knock that area with the feet.

\subsection{COCKPIT ENTRY AND EGRESS}
\textbf{Front Seat Entry} --- Stand on the left wing walk area. 
% Place a towel on the front seat cushion. Place the right foot on the seat cushion. 
Place the right foot on the gold coloured wing spar centre section to the right of the front seat cushion. Support the weight of the upper body by placing the hands on the sides of the cockpit, or on the tubular structure behind the front seat back. Place the left foot on the cockpit floor ahead of the seat cushion.

\textbf{Rear Seat Entry} --- Stand on the left wing walk area. Place a towel on the rear seat cushion. Hang onto the tubular framework just behind the front seat back. Place the right foot on the seat cushion. Place the left foot on the floor ahead of the seat. Place the left hand on the canopy rail. Place the right foot on the floor ahead of the seat. Sit down.

\begin{Note}[CAUTIONS]
  \textbf{Flaps} --- Extend the flaps 90\% for cockpit entry and egress, to ensure that the flaps are not stepped upon.
  
  \textbf{Front Seat} --- Do not use the edge of the windscreen as a handhold when entering or exiting the cockpit. The windscreen fairing that extends past the edge of the
windscreen may be damaged if it is subjected to too much force. Be careful to not kick the intercom system volume knobs on the lower edge of the instrument panel with the feet. Be careful to not kick the alternate static valve on the inboard side of the right landing gear box.

  \textbf{Rear Seat} --- Warn all passengers to be careful not to lean back against the sliding canopy when they enter or leave the aircraft.
  \end{Note}
  
\subsection{STARTING ENGINE}
%follow paragraphs slavishly lifted from the Cessna 177RG POH, as it has a similar engine model.
In cold weather, the engine compartment temperature falls rapidly following engine shutdown, and the injector nozzle lines remain nearly full of fuel. Cold weather starting procedures are therefore relatively simple, with highly predictable results. However, in extremely hot weather, engine compartment temperatures increase rapidly following engine shutdown, and fuel in the lines will vapourize and escape into the intake manifold.

Hot weather starting procedures depend considerably on how soon the next engine start is attempted. Within the first \textcolor{red}{20 to 30 minutes} after shutdown, the fuel manifold is adequately primed, and the empty injector nozzle lines will fill before the engine dies. However, after approximately \textcolor{red}{30 minutes}, the vapourized fuel in the manifold will have nearly dissipated, and some slight ``priming'' could be required to refill the nozzle lines and keep the engine running after the initial start. Starting a hot engine is facilitated by advancing the mixture control promptly to one third open when the engine fires, and then smoothly to full rich as power develops.

Should the engine tend to die after starting, temporarily select the boost pump ON and adjust throttle as necessary to keep the engine running.

In the event of over-priming or flooding, select the boost pump OFF, open the throttle from one half to full open, and continue cranking with the mixture at CUT-OFF. When the engine fires, smoothly advance the mixture control to FULL RICH and retard the throttle to the desired idle speed.

If the engine is under-primed (most likely in cold weather with a cold engine) it will not fire at all, and additional priming will be necessary.

After starting, if the oil pressure gauge does not begin to show oil pressure within 30 seconds in the summertime, and 60 seconds in very cold weather, stop the engine and investigate.

\begin{Note}
  Additional details concerning cold weather starting and operation may be found under COLD WEATHER OPERATION
  paragraphs in this section.
  \end{Note}

\subsection{TAXIING}
\subsubsection{AUTOPILOT}
\textbf{Pre-Flight Test} --- A preflight override and disengagement test should be conducted if the autopilot is to be used in IMC or at night. 

\begin{enumerate}
  \item WING LVLR --- ON (Right Console)
%  \item TURN knob --- Centre
  \item Autopilot --- ON
  \item H~NAV and V~NAV --- Engage
  % \item TK$\bullet$TC$\bullet$WL Switch --- WL. Engages Wing-Leveler mode.
  % \item Make left and right turns during taxi --- Stick should move against the turn.
  \item Controls --- Override. Confirm that the autopilot servo clutches will slip to allow the stick to be moved in both axis when the autopilot is engaged.
  \item Trim/Wing-Leveler Disconnect Switch on control stick --- Press and Release. Confirm both servos release while disconnect switch is pressed. Release the Disconnect switch.
  % \item TK$\bullet$TC$\bullet$WL Switch --- TC. Selects Turn Coordinator mode, disengaging the wing-leveler servo from the flight controls.
  \end{enumerate}

\needspace{15\baselineskip}
  \subsubsection{CONTROL USE DURING TAXIING}
When taxiing, it is important that aileron and elevator be used as appropriate to the wind direction (see Figure \ref{taxi-diagram}. Taxiing over loose gravel should only be done at low engine speed to avoid abrasion and stone damage to propeller tips.
\begin{figure}[htb]
  \begin{center}
  \begin{overpic}[bb=0 0 336 330]{../Diagrams/taxi}
    \newsavebox{\windarrow}
    \newsavebox{\windarroww}
%    \savebox{\windarrow}(10,10){\Line(5,0)}
%    \savebox{\windarrow}(10,10){\drawline(0,0)(5,0)(5,5)(6,5)(2.5,7.5)(-1,5)(0,5)(0,0)}
    \savebox{\windarrow}(10,10){\drawline(0,0)(7,0)(7,-1)(9.5,2.5)(7,6)(7,5)(0,5)(0,0)\put(0.3,1.4){WIND}}
    \savebox{\windarroww}(10,10){\drawline(0,0)(-7,0)(-7,-1)(-9.5,2.5)(-7,6)(-7,5)(0,5)(0,0)\put(-8.4,1.4){WIND}}
%    \arrowlength{12pt}
    \put(5,90){\rotatebox{315}{\usebox{\windarrow}}}
    \put(76,83){\rotatebox{45}{\usebox{\windarroww}}}
    \put(5,7.5){\rotatebox{45}{\usebox{\windarrow}}}
    \put(76,14){\rotatebox{315}{\usebox{\windarroww}}}
    \put(8,70){\begin{boxedminipage}[b]{1.3in}\small \centering USE AFT STICK AND LEFT AILERON\end{boxedminipage}}
    \put(60,70){\begin{boxedminipage}[b]{1.3in}\small \centering USE AFT STICK AND RIGHT AILERON\end{boxedminipage}}
    \put(8,37.5){\begin{boxedminipage}[t]{1.3in}\small \centering USE FORWARD STICK AND RIGHT AILERON\end{boxedminipage}}
    \put(60,37.5){\begin{boxedminipage}[t]{1.3in}\small \centering USE FORWARD STICK AND LEFT AILERON\end{boxedminipage}}
    \end{overpic}
    \end{center}

%  \textcolor{red}{Add diagram of stick position vs wind while taxiing.}
  \caption{Control Use During Taxiing}
  \label{taxi-diagram}
  \end{figure}

\subsection{BEFORE TAKEOFF}
  \subsubsection{WARM-UP}
  Since the engine is closely cowled, care should be taken to avoid over-heating during prolonged engine operation on the ground. Long periods of idling at low RPM may caused fouled spark plugs. It is advisable to lean aggressively while at low power on the ground. If the mixture is leaned, it must be leaned far enough that the engine would run noticeably rough during the higher power setting of the ignition check.
  
  \subsubsection{IGNITION CHECK}
  The ignition check should be done at 1800 RPM as follow:
  \ifthenelse{\thePMAG = 0}{
  \begin{enumerate}
  \item Select the IGNITION, MAG switch to OFF. There should be no RPM drop, as the electronic ignition is more advanced than the magneto at this condition, so the spark from the magneto contributes little. The engine should run smoothly on the electronic ignition.
  \item Select the IGNITION, MAG switch to ON.
  \item Select the IGNITION, ELEC switch to OFF. The RPM drop should be approximately 100 RPM, and the engine should run smoothly on the magneto. 
  \item Select the IGNITION, ELEC switch to ON.
  \end{enumerate}
  }{
  \begin{enumerate}
  \item Select the IGNITION, MAG switch to OFF. There should be approximately 20 rpm drop. The engine should run smoothly on the Light Speed electronic ignition.
  \item Select the IGNITION, MAG switch to ON.
  \item Select the IGNITION, ELEC switch to OFF. The RPM drop should be approximately 70 RPM, and the engine should run smoothly on the PMag electronic ignition. 
  \item Pull the PMag CB, to force the PMag to rely on its own internal alternator to provide ignition power, confirming proper operation on this power source.
  \item Reset the PMag CB.
  \item Select the IGNITION, ELEC switch to ON.
  \end{enumerate}
  }
  
  \subsubsection{ELECTRICAL SYSTEM CHECK}
  The following should be conducted prior to night or IFR flight:
  \begin{enumerate}
    \item Select ESS BUS FEED switch to EMER.
    \item Select BATT/ALT switch to OFF. The low voltage condition should cause the``ENGINE MONT.'' light to flash, and the EIS 4000 should switch to a page showing the voltage. The GNS 430W should still be operating, indicating that the Essential Bus is powered.
    \item Select STBY ALT switch to ON. 
    \item Increase engine speed to 1800 rpm.
    \item Check the Voltage --- it should be approximately 12.8 volts, indicating that the standby alternator is operative.
    \item Select STBY ALT switch to OFF
    \item Select BATT/ALT switch to BATT + ALT
    \item Select ESS BUS FEED switch to NORM.
    \end{enumerate}
  
\subsection{STABILITY AND CONTROL}
  \subsubsection{STATIC LONGITUDINAL STABILITY}
  The aircraft has an unusually large range between the forward and aft CG limits, which leads to substantial differences in stability and control characteristics between forward and aft CG loadings. At forward CG, the stick forces are relatively high, and the aircraft has strong static longitudinal stability. The stick forces required for a given manoeuvre decrease significantly as the CG moves aft, and the static longitudinal stability becomes much weaker. 
  
The low speed, power off, stick free static longitudinal stability is neutral to slightly negative at the aerobatic aft CG limit. If the aircraft is trimmed at a particular speed, and the speed is changed, the aircraft has no natural tendency to return to the trimmed speed. 
 
The stick free static longitudinal stability is degraded as engine power is increased, and becomes noticeably negative at high power at low speed at aft CG. If the aircraft is trimmed for a climb speed, and the speed is reduced, a push force is needed to stabilize at a lower speed, and the speed will reduce to the stall if the stick is released. At higher speeds, as in for cruise or descent, the stick free static longitudinal stability is positive, even at aft CG and high power.
 
\begin{Note}[WARNING]
The aircraft will diverge from the trimmed speed at high power at low speed at aft CG. It will decelerate to the stall if left to its own devices. Particular attention should be paid to airspeed control during climbs in IMC conditions, or at climbs at low altitude in any weather conditions.
\end{Note}

\subsection{TAKEOFF}
  \subsubsection{POWER CHECK}
    It is important to check full-throttle engine operation early in the takeoff run. Any sign of rough engine operation or sluggish engine acceleration is good cause for rejecting the takeoff.
    
    Full throttle operation over loose gravel is especially harmful to propeller tips. When takeoffs must be made over a gravel surface, it is very important that the throttle be advanced slowly, to allow the aircraft to start rolling before high RPM is developed. The tail should be kept down longer than normal to increase the propeller ground clearance.
    
    Prior to takeoffs from airfields above \textcolor{red}{3,000} feet elevation, the mixture should be leaned to the value on the placard located ahead of the throttle quadrant.
    
  \subsubsection{FLAP SETTING}
    Normal takeoffs are performed with flaps retracted. Short field takeoffs are performed with one-third flap, which is set by deflecting the ailerons fully, and extending the flaps so that the flap angle matches the angle of the down aileron.
    
  \subsubsection{CROSSWIND TAKEOFF}
    Takeoffs have been demonstrated with crosswinds of \textcolor{red}{12} knots from the left and \textcolor{red}{8} knots from the right, with flaps retracted. 

\subsection{CLIMB}
Note the highest EGT on the bottom right corner of the EIS 4000 default display page immediately after selecting climb power following take-off. Reduce the mixture as required during climb to keep the highest EGT close to this value without exceeding it.

If there are obstacles to clear, climb at $\mathrm{V_{X}}$ (\textcolor{red}{65} KIAS at sea level and \textcolor{red}{77} KIAS at 10,000 ft). Use full throttle and 2700 RPM, with mixture adjusted to maintain the target EGT and flaps retracted.

If maximum rate of climb is needed, use the climb speed from Figure \ref{ROC-Max}, full throttle and 2700 RPM, with mixture adjusted to maintain the target EGT and flaps retracted.

Normal climbs should be made at the speed from Figure \ref{TFD-to-climb-Norm}, using full throttle and 2700 RPM.
\subsection{CRUISE}
Normal cruise power is between 55\% and 75\% power. Adjust the mixture to achieve the fuel flow given in Figure \ref{Cruise-power}. Adjust the oil cooler door to obtain an oil temperature of approximately 190\textdegree F.
\subsection{STALLS AND SPINS}

\subsubsection{STALLS} The aircraft has very little natural stall warning in wings level and 30\textdegree \ bank stalls. There is a small amount of buffet approximately one kt prior to the stall. The nose drops at the stall. A small wing drop may occur if the ball is not centred. There is a sharp left wing drop at stalls with high power settings.

There is significant, progressively increasing aerodynamic buffet prior to stalls with 2g or higher load factor.

\subsubsection{SPINS}\textcolor{red}{Complete this section following flight test. Add info on susceptibility to inadvertent spin, intentional spin entry procedure, spin recovery procedure, altitude loss during spin recovery and any relevant info on inverted spins.}

There is insufficient nose up elevator authority to keep in the aircraft in a spin at mid or forward CG with the power at idle. The angle of attack, pitch rate and roll rate are oscillatory. If pro-spin controls are held, the IAS, load factor and rotation rate increase, indicating a spiral dive. One wing may stall during one of the pitch rate oscillations, leading to a very sharp snap roll.

\begin{Note}[Caution]
If the IAS increases during a spin, recover immediately, as the aircraft is in a spiral dive. If pro-spin controls are held, the aircraft may do a violent high g snap roll which could damage the structure. 
\end{Note}

\subsection{AEROBATICS}

Refer to manoeuvring speed and weight and balance limitations when
contemplating aerobatics. The manoeuvring speed is the highest speed at
which full and abrupt control can be applied without exceeding design
loads. This is not the highest permissible aerobatic entry speed, but
control inputs must be limited to less than full at any speed above
manoeuvring speed.

The entry speeds for some manoeuvres can vary over a wide range due
to the large ratio of maximum speed to stall speed. For vertical manoeuvres
(e.g. loops, Immelmann turns and Cuban eights) entry speed has
an inverse relationship to G forces required to complete the manoeuvre.
An entry speed at lower speeds will require a higher G pull up than
for entry near top end of speed range. 

\begin{Note}[WARNING]
Excessive speed build up can occur very quickly, particularly in a
dive. The RV-8 is a pilot limited aircraft due to the light control
forces and aerodynamic cleanliness --- it is the pilot's responsibility
 to not overstress the aircraft. The stick forces vary considerably
with CG position --- stick forces at aft CG are much lighter then stick
forces at forward CG.
\end{Note}

\begin{Note}
The following list of manoeuvres are approved 
when operating at weights less than 703.1 kg (1550 lb) and with the
CG in the aerobatic envelope. 
\end{Note}

\begin{center}\begin{tabular}{lc}
Manoeuvre&
Entry Speed\tabularnewline
&
(KIAS)\tabularnewline
Upright spins&55 (at stall)\tabularnewline
Loops, Cuban eights&
100 -- 180\tabularnewline
Immelmann turns&
130 -- 180\tabularnewline
Aileron Rolls, Barrel rolls&
105 -- 165\tabularnewline
Hammerheads&
155 -- 165\tabularnewline
% Snap Rolls&
% 70 -- 95\tabularnewline
Vertical rolls&
155 -- 165\tabularnewline
Split--S&
85 -- 120\tabularnewline
\end{tabular}\end{center}

\subsection{DESCENT}
Close the oil cooler door during descent to keep the oil warm.
\begin{Note} 
Attempt to manage power and airspeed during descent to keep the CHT cooling rate no greater than 50\textdegree F/min to avoid shock cooling, preferably 30\textdegree F/min or less.
\end{Note}


\subsection{BEFORE LANDING}

\subsection{LANDING}
Landings may be made with any flap angle. Landings in strong crosswind should be made with flaps retracted. Landings have been demonstrated with winds of 20 knots from the left and 25 knots from the right, with flaps retracted. Wheel landings are preferred, but three-point landings may be made if a shorter landing distance is required.

\begin{Note}[CAUTION]
The tailwheel steering unlocks at full left rudder, which may lead to a ground loop in strong left crosswind.\end{Note}

\subsection{SHUTDOWN}
\begin{Note}[CAUTION]
With the flaps fully extended, there is a risk that someone standing behind the wing could press on the flap trailing edge, and force the flaps to move far enough that the forward edge of the upper flap skin could move aft of the aft edge of the upper wing skin. If this condition is not detected significant flap and/or upper wing skin damage may occur when the flaps are retracted. To reduce the probability of this happening, the flaps should not be left in the fully extended position on the ground. Instead, leave them approximately 90\% extended.\end{Note}

\needspace{10\baselineskip}
\subsection{COLD WEATHER OPERATION}
\subsubsection{Engine Preheat} The engine should be preheated prior to start if the temperature has been below 0\textdegree C. The Reiff preheater plug is clamped to the back of the rear baffle and can be accessed through the oil filler door. An engine blanket is required to allow the 300W preheat to increase the engine temperature to adequate levels if the temperature is -15\textdegree C or colder.

\subsubsection{Wheel Pants} The wheel pants must be removed for operations in loose snow, as snow will collect in the pants. The heat from the brakes will melt snow and the resulting water will later freeze on the brakes.

\subsection{HOT WEATHER OPERATION}
\textcolor{red}{Hot weather operations section to be added.}

\subsection{AUTOPILOT}
\subsubsection{Altimeter Setting} The autopilot does not have a means to directly enter an altimeter setting. Instead, the current altitude is entered whenever the "BARO SET" message is displayed, which allows the system to determine the needed offset to the pressure altitude sensed from the static input. The "BARO SET" message will appear following initial power up and any time a selected altitude is entered.

The autopilot will normally engage in ALT HLD mode, with the current altitude as the reference. The ALT HLD reference altitude may be adjusted by pressing the "V~MODE" button until "ALT ADJ UP/DN" is displayed, then turning the rotary knob. CW rotation increases the altitude by 5 ft/click, and CCW rotation decreases it by 5 ft/click. Alternatively, the red AP/TRIM Disconnect switch on the control stick may be pressed and held for a minimum of 5 seconds, which releases the servos and allows the pilot to manoeuvre to the desired altitude, which resets the ALT HLD reference.

\subsubsection{ALT HLD Performance} ALT HLD mode works well in straight and level flight, but up to 200 ft altitude will be lost in turns at 30\textdegree \ bank at forward CG. Performance at aft CG is much better, with 50 ft of altitude loss typically seen. The pitch servo should be disengaged and the altitude controlled manually during turns with more than 30\textdegree \ of heading change when flying IFR at forward CG. Alternatively, when following radar vectors, the red AP/TRIM Disconnect switch on the control stick may be pressed and held while the aircraft is manually manoeuvred to the new heading. The current course and altitude will be maintained once the disconnect switch is released.

\begin{Note}
  The autopilot lateral mode will switch to CRS when the red AP/TRIM Disconnect switch on the control stick is pressed and held.
\end{Note}
\cleardoublepage
