% EFIS ASI TAS Correction Table
\begin{figure}[t]
% \addcontentsline{toc}{section}{Figure \ref{TO-Dist} Takeoff Distance}
\addcontentsline{toc}{section}{EFIS ASI TAS CORRECTION}
\begin{center}
\begin{perfhdr}EFIS ASI TAS CORRECTION\\
1800 LBS
\end{perfhdr}
\Large
% \textcolor{red}{VANS CLAIMED PERF EXPANDED TO OTHER CONDITIONS}\vspace{1ex}\\
% \textcolor{red}{TO BE CONFIRMED BY FLIGHT TEST}\normalsize \vspace{5ex}\\

% \begin{minipage}{7.5in}
%   \begin{flushleft}
%     CONDITIONS:\\
%     Flaps 17\textdegree \ (set flap angle to match down aileron angle at full aileron)\\
%     2700 RPM, Full Throttle and Mixture Set prior to Brake Release\\
%     Paved, Level, Dry Runway\\
%     Zero Wind\\
% \vspace{\perfnoteskip}
%     NOTES:
%     \begin{enumerate*}
%       \item Short field technique as specified in Section \textcolor{red}{4}.
% %      \item Prior to takeoff from fields above 8,000 ft elevation, the mixture should be leaned to give maximum power in
% %      a full throttle static runup.
%       \item Set mixture at placard fuel flow.
%       \item Decrease distance by 10\% for each \textcolor{red}{X} knots headwind.  For operations with tailwinds up to 10
%       knots, increase distances by \textcolor{red}{10\%}.
%       \item For operation on a dry, grass runway, increase distances by \textcolor{red}{10\%} of the ground roll figure.
%       \end{enumerate*}
%     \end{flushleft}
%   \end{minipage}
% \hfill
% \begin{minipage}{1.5in}
%   \begin{tabular}{|c|c|}
%     \hline
%     \multicolumn{2}{|c|}{MIXTURE SETTING}\\
%     \hline
%     PRESS ALT&GPH\\
%     \hline
%     S.L.&17\\
%     2000&16\\
%     4000&15\\
%     6000&14\\
%     8000&13\\
%     \hline
%     \end{tabular}
%   \end{minipage}
% \\
\vspace{\perfnoteskip}
    \raggedright NOTES:
    \begin{enumerate*}
      \item Table provides TAS correction as a function of altitude and IAS.
      \item Corrected TAS = TAS displayed on EFIS + correction.
      \item Table includes static source position error, EFIS ASI instrument error and OAT ram temperature rise.
      \end{enumerate*}

\vspace{\perfnoteskip}
\settowidth{\colOne}{ALTITUDE}
% \settowidth{\colFive}{GRND}
\begin{tabular}{|c|c|c|c|c|c|c|c|c|c|c|c|}
\hline
\multirow{2}{\colOne}{\centering ALTITUDE (FT)}&\multicolumn{11}{c|}{IAS (KT)}\\
\cline{2-12}
&80&90&100&110&120&130&140&150&160&170&180\\
\hline
\hline
0&+1.2&+1.0&+0.8&+0.4&+0.0&-0.5&-0.7&-1.3&-2.1&-2.9&-3.6\\
\hline
5,000&+1.3&+1.1&+0.8&+0.4&-0.0&-0.6&-0.9&-1.5&-2.4&-3.2&-4.0\\
\hline
10,000&+1.3&+1.1&+0.8&+0.4&-0.1&-0.7&-1.0&-1.7&-2.7&-3.6&-4.5\\
\hline
15,000&+1.4&+1.2&+0.9&+0.4&-0.2&-0.8&-1.3&-2.0&-3.1&-4.2&-5.1\\
\hline
20,000&+1.5&+1.3&+0.9&+0.3&-0.3&-1.1&-1.5&-2.4&-3.6&-4.8&-5.8\\
\hline
&\multicolumn{11}{c|}{TAS CORRECTION (KT)}\\
\hline
\end{tabular}
\end{center}
\caption{EFIS ASI TAS Correction}
\label{EFIS-ASI-TAS-Corr}
\end{figure}

