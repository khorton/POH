% Time, Fuel and Distance to Climb table --- Normal Climb Power
\begin{figure}[t]
% \addcontentsline{toc}{section}{Figure \ref{TFD-to-climb-Norm} Time, Fuel and Distance to Climb --- Normal Climb Power}
\addcontentsline{toc}{section}{TIME, FUEL AND DISTANCE TO CLIMB --- MAX EFFICIENCY CLIMB --- 1900 lb}
\begin{center}
\begin{perfhdr}TIME, FUEL AND DISTANCE TO CLIMB --- 1900 lb\\
MAX EFFICIENCY CLIMB\\
\end{perfhdr}
\Large

\normalsize
\vspace{5ex}
    \raggedright 
    CONDITIONS:\\
    Flaps UP\\
    2650 RPM\\
    Full Throttle\\
    Mixture Set to give Take-off EGT\\
    Standard Temperature\\
\hfill

\vspace{\perfnoteskip}
    \raggedright NOTES:
    \begin{enumerate*}
      \item Add 1.5 USG of fuel for engine start, taxi and takeoff.
%      \item Mixture leaned when power is at 75\% power or less (6,000 ft or above at standard temperature) for smooth engine operation and increased power.
      \item Climb speed is 130 KIAS, until the rate of climb reduces to 500 ft/mn.  Then hold 500 ft/mn until the speed reduces to $\mathrm{V_{Y}}$.
      \item Increase time, fuel and distance by 10\% for each 10\textdegree C above standard temperatures.
      \item Distances shown are based on zero wind.
      \end{enumerate*}
\vspace{\perfnoteskip}
% following lengths are used to set row widths to fit the text
\settowidth{\colOne}{WEIGHT}
\settowidth{\colTwo}{PRESSURE}
\settowidth{\colThree}{TEMP}
\settowidth{\colFour}{CLIMB}
\settowidth{\colFive}{RATE OF}
\settowidth{\colSix}{TIME}
\settowidth{\colSeven}{USED}
\settowidth{\colEight}{DIST.}

\begin{tabular}{|c|r|r|r|r|r|r|r|}
\hline
\multirow{3}{\colOne}[\halfrowdrop]{\centering WEIGHT (LB)}&\multirow{3}{\colTwo}[\halfrowdrop]{\centering PRESSURE ALTITUDE (FT)}&
\multirow{3}{\colThree}[\halfrowdrop]{\centering TEMP (\textdegree C)}&\multirow{3}{\colFour}[\halfrowdrop]{\centering CLIMB SPEED (KIAS)}&
\multirow{3}{\colFive}[\halfrowdrop]{\centering RATE OF CLIMB (FT/MN)}&\multicolumn{3}{c|}{FROM SEA LEVEL}\\
\cline{6-8}
&&&&&\multicolumn{1}{m{\colSix}|}{\centering TIME (MN)}&\multicolumn{1}{m{\colSeven}|}{\centering FUEL USED (USG)}&\multicolumn{1}{m{\colEight}|}{\centering DIST. (NM)}\\
\hline
\hline
% % START of copied data from sage script

1,900&0&15&130&1,560&0&0&0\\
\hline
&2,000&11&130&1,310&1&0.4&3\\
\hline
&4,000&7&130&1,090&3&0.8&7\\
\hline
&6,000&3&130&870&5&1.4&12\\
\hline
&8,000&-1&130&650&8&2.0&18\\
\hline
&10,000&-5&127&500&11&2.8&27\\
\hline
&12,000&-9&115&500&15&3.6&36\\
\hline
&14,000&-13&96&500&19&4.4&45\\
\hline
&16,000&-17&90&370&24&5.2&53\\
\hline
&18,000&-21&90&230&30&6.3&66\\
\hline
&20,000&-25&90&90&43&8.3&92\\
\hline
% % END of copied data from sage script
\end{tabular}
\end{center}
\caption{Time, Fuel and Distance to Climb --- Max Efficiency Climb --- 1900 lb}
\label{TFD-to-climb-Norm}
\end{figure}


