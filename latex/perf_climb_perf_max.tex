% Time, Fuel and Distance to Climb table --- Max Power
\begin{figure}[t]
% \addcontentsline{toc}{section}{Figure \ref{TFD-to-climb-Max} Time, Fuel and Distance to Climb --- Maximum Power}
\addcontentsline{toc}{section}{TIME, FUEL AND DISTANCE TO CLIMB --- MAXIMUM POWER}
\begin{center}
\begin{perfhdr}TIME, FUEL AND DISTANCE TO CLIMB\\
MAXIMUM POWER\\
\end{perfhdr}
\Large
\textcolor{red}{DATA TO BE CONFIRMED BY FLIGHT TEST}\normalsize \vspace{5ex} \\
\begin{minipage}{4in}
  \begin{flushleft}
    CONDITIONS:\\
    Flaps UP\\
    2700 RPM\\
    Full Throttle\\
    Mixture Set at Placard Fuel Flow
    Standard Temperature\\

    \end{flushleft}
\end{minipage}
\hfill
\begin{minipage}{1.5in}
  \begin{tabular}{|c|c|}
    \hline
    \multicolumn{2}{|c|}{MIXTURE SETTING}\\
    \hline
    PRESS ALT&GPH\\
    \hline
    S.L.&17\\
    2000&16\\
    4000&15\\
    6000&14\\
    8000&13\\
    \hline
    \end{tabular}
  \end{minipage}
\\
\vspace{\perfnoteskip}
    \raggedright NOTES:
    \begin{enumerate*}
      \item Add 1.0 USG of fuel for engine start, taxi and takeoff.
%      \item Mixture leaned when power is at 75\% power or less (8,000 ft or above at standard temperature) for smooth
%      engine operation and increased power.
      \item Climb speed is 100 KIAS at sea level, decreasing by 1 kt per 1000 ft.
      \item Increase time, fuel and distance by \textcolor{red}{XX\%} for each 10\textdegree C above standard
      temperatures.
      \item Distances shown are based on zero wind.
      \end{enumerate*}

\vspace{\perfnoteskip}
% following lengths are used to set row widths to fit the text
\settowidth{\colOne}{WEIGHT}
\settowidth{\colTwo}{PRESSURE}
\settowidth{\colThree}{TEMP}
\settowidth{\colFour}{CLIMB}
\settowidth{\colFive}{RATE OF}
\settowidth{\colSix}{TIME}
\settowidth{\colSeven}{USED}
\settowidth{\colEight}{DIST.}

\begin{tabular}{|c|r|r|r|r|r|r|r|}
\hline
\multirow{3}{\colOne}[\halfrowdrop]{\centering WEIGHT (LB)}&\multirow{3}{\colTwo}[\halfrowdrop]{\centering PRESSURE ALTITUDE (FT)}&
\multirow{3}{\colThree}[\halfrowdrop]{\centering TEMP (\textdegree C)}&\multirow{3}{\colFour}[\halfrowdrop]{\centering CLIMB SPEED (KIAS)}&
\multirow{3}{\colFive}[\halfrowdrop]{\centering RATE OF CLIMB (FT/MN)}&\multicolumn{3}{c|}{FROM SEA LEVEL}\\
\cline{6-8}
&&&&&\multicolumn{1}{m{\colSix}|}{\centering TIME (MN)}&\multicolumn{1}{m{\colSeven}|}{\centering FUEL USED (USG)}&\multicolumn{1}{m{\colEight}|}{\centering DIST. (NM)}\\
\hline
\hline
% % START of copied data from python perf script
% % Data use 95% of the calculated power to match the results from the flight test with the test the Hartzell prop
1800&0&15&100&1870&0&0&0\\
\hline
&2,000&11&98&1700&1&0.3&2\\
\hline
&4,000&7&96&1530&2&0.6&4\\
\hline
&6,000&3&94&1370&4&0.9&6\\
\hline
&8,000&-1&92&1210&5&1.2&9\\
\hline
&10,000&-5&90&1050&7&1.5&12\\
\hline
&12,000&-9&88&910&9&1.9&16\\
\hline
&14,000&-13&86&760&12&2.3&20\\
\hline
&16,000&-17&84&620&14&2.8&25\\
\hline
&18,000&-21&82&480&18&3.4&32\\
\hline
&20,000&-25&80&340&23&4.1&41\\
\hline
% % END of copied data from python perf script
\end{tabular}
\end{center}
\caption{Time, Fuel and Distance to Climb --- Maximum Power}
\label{TFD-to-climb-Max}
\end{figure}



