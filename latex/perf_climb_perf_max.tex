% Time, Fuel and Distance to Climb table --- Max Power
\begin{figure}[t]
% \addcontentsline{toc}{section}{Figure \ref{TFD-to-climb-Max} Time, Fuel and Distance to Climb --- Maximum Power}
\addcontentsline{toc}{section}{TIME, FUEL AND DISTANCE TO CLIMB --- MAXIMUM POWER}
\begin{center}
\begin{perfhdr}TIME, FUEL AND DISTANCE TO CLIMB\\
MAXIMUM POWER\\
\end{perfhdr}
\Large
% \textcolor{red}{DATA TO BE CONFIRMED BY FLIGHT TEST}\\
\normalsize \vspace{5ex} 
\raggedright 
    CONDITIONS:\\
    Flaps UP\\
    2650 RPM\\
    Full Throttle\\
    Mixture Set to give EGT 25\textdegree F less than EGT during take-off\\
    Standard Temperature\\

\hfill

\vspace{\perfnoteskip}
    \raggedright NOTES:
    \begin{enumerate*}
      \item Add 1.0 USG of fuel for engine start, taxi and takeoff.
%      \item Mixture leaned when power is at 75\% power or less (8,000 ft or above at standard temperature) for smooth
%      engine operation and increased power.
      \item Climb speed is 102 KIAS at sea level, decreasing by 1 kt per 1000 ft.
      \item Increase time, fuel and distance by 10\% for each 10\textdegree C above standard
      temperatures.
      \item Distances shown are based on zero wind.
      \end{enumerate*}

\vspace{\perfnoteskip}
% following lengths are used to set row widths to fit the text
\settowidth{\colOne}{WEIGHT}
\settowidth{\colTwo}{PRESS.}
\settowidth{\colThree}{TEMP}
\settowidth{\colFour}{CLIMB}
\settowidth{\colFive}{RATE OF}
\settowidth{\colSix}{TIME}
\settowidth{\colSeven}{USED}
\settowidth{\colEight}{DIST.}

\begin{tabular}{|c|r|r|r|r|r|r|r|}
\hline
\multirow{3}{\colOne}[\halfrowdrop]{\centering WEIGHT (LB)}&\multirow{3}{\colTwo}[\halfrowdrop]{\centering PRESS. ALT. (FT)}&
\multirow{3}{\colThree}[\halfrowdrop]{\centering TEMP (\textdegree C)}&\multirow{3}{\colFour}[\halfrowdrop]{\centering CLIMB SPEED (KIAS)}&
\multirow{3}{\colFive}[\halfrowdrop]{\centering RATE OF CLIMB (FT/MN)}&\multicolumn{3}{c|}{FROM SEA LEVEL}\\
\cline{6-8}
&&&&&\multicolumn{1}{m{\colSix}|}{\centering TIME (MN)}&\multicolumn{1}{m{\colSeven}|}{\centering FUEL USED (USG)}&\multicolumn{1}{m{\colEight}|}{\centering DIST. (NM)}\\
\hline
\hline
% % START of copied data from python Climb_DA_fit.py script
% % Based on climb tests on flights 261, 262, 263 & 264
1,800&0&15&102&1,810&0&0&0\\
\hline
&2,000&11&100&1,640&1&0.3&2\\
\hline
&4,000&7&98&1,470&2&0.7&4\\
\hline
&6,000&3&96&1,300&4&1.0&7\\
\hline
&8,000&-1&94&1,130&6&1.4&10\\
\hline
&10,000&-5&92&960&7&1.8&13\\
\hline
&12,000&-9&90&790&10&2.3&17\\
\hline
&14,000&-13&90&620&13&2.9&22\\
\hline
&16,000&-17&90&450&16&3.5&29\\
\hline
&18,000&-21&90&290&22&4.5&40\\
\hline
&20,000&-25&90&120&32&6.1&60\\
\hline
% % END of copied data from python Climb_DA_fit.py script
\end{tabular}
\end{center}
\caption{Time, Fuel and Distance to Climb --- Maximum Power}
\label{TFD-to-climb-Max}
\end{figure}



