% Time, Fuel and Distance to Climb table --- Normal Climb Power
\begin{figure}[t]
% \addcontentsline{toc}{section}{Figure \ref{TFD-to-climb-Norm} Time, Fuel and Distance to Climb --- Normal Climb Power}
\addcontentsline{toc}{section}{TIME, FUEL AND DISTANCE TO CLIMB --- NORMAL CLIMB POWER}
\begin{center}
\begin{perfhdr}TIME, FUEL AND DISTANCE TO CLIMB\\
NORMAL CLIMB\\
\end{perfhdr}
\Large
\textcolor{red}{DATA TO BE CONFIRMED BY FLIGHT TEST}\normalsize \\
\vspace{5ex}
\begin{minipage}{4in}
  \begin{flushleft}
    CONDITIONS:\\
    Flaps UP\\
    2500 RPM\\
    Full Throttle\\
    Mixture Set at Indicated Fuel Flow
    Standard Temperature\\
    \end{flushleft}
\end{minipage}
\hfill
\begin{minipage}{1.5in}
  \begin{tabular}{|c|c|}
    \hline
    \multicolumn{2}{|c|}{MIXTURE SETTING}\\
    \hline
    PRESS ALT&GPH\\
    \hline
    S.L. to 4,000&13\\
    8,000&11\\
    12,000&8\\
    \hline
    \end{tabular}
  \end{minipage}
\\
\vspace{\perfnoteskip}
    \raggedright NOTES:
    \begin{enumerate*}
      \item Add 1.5 USG of fuel for engine start, taxi and takeoff.
%      \item Mixture leaned when power is at 75\% power or less (6,000 ft or above at standard temperature) for smooth engine operation and increased power.
      \item Climb speed is 100 KIAS from sea level to 10,000 ft, then decreasing by 2 kt per 1000 ft above 10,000 ft.
      \item Increase time, fuel and distance by \textcolor{red}{XX\%} for each 10\textdegree C above standard temperatures.
      \item Distances shown are based on zero wind.
      \end{enumerate*}
\vspace{\perfnoteskip}
% following lengths are used to set row widths to fit the text
\settowidth{\colOne}{WEIGHT}
\settowidth{\colTwo}{PRESSURE}
\settowidth{\colThree}{TEMP}
\settowidth{\colFour}{CLIMB}
\settowidth{\colFive}{RATE OF}
\settowidth{\colSix}{TIME}
\settowidth{\colSeven}{USED}
\settowidth{\colEight}{DIST.}

\begin{tabular}{|c|r|r|r|r|r|r|r|}
\hline
\multirow{3}{\colOne}[\halfrowdrop]{\centering WEIGHT (LB)}&\multirow{3}{\colTwo}[\halfrowdrop]{\centering PRESSURE ALTITUDE (FT)}&
\multirow{3}{\colThree}[\halfrowdrop]{\centering TEMP (\textdegree C)}&\multirow{3}{\colFour}[\halfrowdrop]{\centering CLIMB SPEED (KIAS)}&
\multirow{3}{\colFive}[\halfrowdrop]{\centering RATE OF CLIMB (FT/MN)}&\multicolumn{3}{c|}{FROM SEA LEVEL}\\
\cline{6-8}
&&&&&\multicolumn{1}{m{\colSix}|}{\centering TIME (MN)}&\multicolumn{1}{m{\colSeven}|}{\centering FUEL USED (USG)}&\multicolumn{1}{m{\colEight}|}{\centering DIST. (NM)}\\
\hline
\hline
% % START of copied data from python perf script
% % Data use 95% of the calculated power to match the results from the flight test with the test the Hartzell prop
1800&0&15&100&1630&0&0&0\\
\hline
&2,000&11&100&1480&1&0.3&2\\
\hline
&4,000&7&100&1320&3&0.6&5\\
\hline
&6,000&3&100&1170&4&0.9&8\\
\hline
&8,000&-1&100&1020&6&1.2&11\\
\hline
&10,000&-5&100&870&8&1.6&15\\
\hline
&12,000&-9&96&750&11&2.0&20\\
\hline
&14,000&-13&92&620&14&2.5&25\\
\hline
&16,000&-17&88&500&17&3.0&32\\
\hline
&18,000&-21&84&380&22&3.7&41\\
\hline
&20,000&-25&80&250&28&4.5&52\\
\hline
% % END of copied data from python perf script
\end{tabular}
\end{center}
\caption{Time, Fuel and Distance to Climb --- Normal Climb Power}
\label{TFD-to-climb-Norm}
\end{figure}


